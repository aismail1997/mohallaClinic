\section{Methodology}

\begin{comment}
This should tell the story of what you did, and the single most important point is to show that you were reflective, rigorous, ethical, sound in your research. 
Start with where you conducted the study. Was there one site or many? How long was your (the authors’) engagement with this site? What was the nature of the engagement? And remember to anonymize by default.
Say what methods you used and who you studied. How long were your interviews? What kinds of questions did you ask? Did you use an interpreter? Who were these people? How old?
Add a paragraph on who the authors are and what they bring to the table. Say what their biases might have been that could be limitations of this work. Was there access you could not get? 
How did you analyze the data? If you used grounded theory, whose version of grounded theory did you use? Who was responsible for the analysis? 
Finally, there are lots of papers that have a decent write-up for methodology. Find a researcher you like and read up on the Methodology section.
\end{comment}

\textcolor{red}{}

The study was conducted in urban(?) Delhi following a qualitative approach using interviews and observations. The study was limited to a region of Delhi (should mention migrants etc?) and was conducted in English, Hindi and Urdu.

We interacted with doctors, staff and patients in private and government clinics to obtain their perspective in the process of accessing health care. Each interview lasted around 30 minutes with those patients often lasting longer. 

Outside the clinical settings, we interacted with ASHAs and potential users of the health care system in an attempt to build a more complete picture of health care access in this region. These ASHAs worked in slums or 'jhuggis' and we accompanied the ASHAs during their field visits. The interviews in these settings often lasted over an hour and were a minimum of 30 minutes.

****
Settings:
Government dispensary (MCD)
2 Mohalla clinics
Places visited with ASHAs
slums near the Yamuna
Flats
Pvt - 2 small clinics and a hospital (Al - Shifa)
Slum under a flyover

Interviews with doctors, staff, patients and observations in clinics + accompanied ASHA workers
Audio + interview transcripts + field notes
Ppl: migrants from UP, Bihar, Bengal
Hindi/Urdu/English