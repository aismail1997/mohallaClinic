\section{Methodology}

\begin{comment}
This should tell the story of what you did, and the single most important point is to show that you were reflective, rigorous, ethical, sound in your research. 
Start with where you conducted the study. Was there one site or many? How long was your (the authors’) engagement with this site? What was the nature of the engagement? And remember to anonymize by default.
Say what methods you used and who you studied. How long were your interviews? What kinds of questions did you ask? Did you use an interpreter? Who were these people? How old?
Add a paragraph on who the authors are and what they bring to the table. Say what their biases might have been that could be limitations of this work. Was there access you could not get? 
How did you analyze the data? If you used grounded theory, whose version of grounded theory did you use? Who was responsible for the analysis? 
Finally, there are lots of papers that have a decent write-up for methodology. Find a researcher you like and read up on the Methodology section.
\end{comment}
%\textcolor{red}{[Make this flow. Content is here]}

The multi-site short-term ethnographic study was conducted in urban Delhi from May to July 2016, through a process of semi-structured interviews and observations. The data collected was in the form of audio recordings, field notes and interview transcripts. Interview questions revolved around understanding the nature of the interactions with [workers vs actors] (frontline + experts) in the health care system, engagement with the facilities available and perception of health and health care.

Our study was limited to two sites around six kilometers apart. One of these locations was a slum under a flyover at the intersection of major roads and the researcher's engagement with this site lasted a week. The other site was a market area of a radius of about 500 meters and the research was conducted in multiple settings in this area. 
The researcher conducting the interviews and observations is intimately familiar with this area and we are aware of the preconceived notions and biases this brings to the table, albeit with a deeper understanding of this region.
% all of the data collected by first author - spoke fluently with them in native language  
The population studied was comprised primarily of migrants from Bihar, Uttar Pradesh and Bengal 
%(mostly Muslim?)
and were between 18 to 68 years old. The researchers communicated with the subjects in the colloquial language which is a mix of Hindi, Urdu, and English.

The settings studied included clinical settings - 3 private clinics, 2 Mohalla Clinics, a government dispensary and non-clinical settings - slums, in an effort to paint a more complete picture of the health care system in Delhi. During the course of the study, we interacted with 5 doctors, 4 staff members, 5 front-line health workers (4 ASHAs and 1 ANM) and 50 potential users. Interviews in clinical settings typically lasted between 15-30 minutes though some interactions lasted over an hour. Interviews conducted outside the clinical setting lasted around an hour. Engagement with the front-line health workers and potential users also involved participant observation where we accompanied 2 ASHAs and an ANM during their field work in slums along the Yamuna river for three days.

%coding? - how did we do analysis?
We used Charmaz's grounded theory 
% inductive analysis - x aspect of Charmaz's approach - write memos  
to analyze our data. Data collected was analyzed by the researcher conducting the interviews and observations. [add stuff about coding and analysis - ]
% !!!!!!!
Despite the attempt to paint a complete picture, an important missing piece of our study is the top-down perspective. We were unable to gain the government's perspective during this study. Our knowledge about their goals, ideas and future plans is obtained from press releases, news articles and popular media. 

%is it - change this based on sources we actually use