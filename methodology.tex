\section{Methodology}
We conducted a multisited field study in an urban but low-to-middle income neighborhood in Delhi (India) from May to July 2016. We conducted extensive non-participant and participant observations, in addition to roughly 75 semi-structured interviews, collecting data in the form of audio recordings, photographs, and field notes. All data was collected, transcribed, and translated by the first author who is fluent in the local dialect and intimately familiar with the geography. 

The sites we studied included three private clinics, two Mohalla clinics, a government dispensary, and X slums in the neighborhood of these healthcare settings - all in X area of Delhi. In this note, we draw on our findings from the two Mohalla clinics and the slums. The clinics were approximately six kilometers apart. ???One of these locations was a slum under a flyover at the intersection of major roads and the researcher's engagement with this site lasted a week. The other site was a market area of a radius of about 500 meters and the research was conducted in multiple settings in this area???.

For this note, we draw on our interviews with five doctors, four staff, five frontline health workers, and roughly 50-60 individuals. This group of individuals included members of the target population (some were users while others were non-users of the Mohalla clinics) as well as users of the clinics who did not fall under the target population. Many among the targeted communities were migrants from the neighboring states of Bihar, Uttar Pradesh, and Bengal. Most participants were Muslim and ranged between 18 to 68 years old. All came from roughly the same neighborhood that constituted the surroundings of the clinics. The interviews in the clinics lasted 15-30 minutes in general, though some went for over an hour. Interviews outside the clinics lasted approximately an hour. Questions we asked in these interviews focused on the interactions between different actors in the healthcare system, ranging from patients to workers to experts, also inquiring about their engagement with the system (or lack thereof), and the factors that influenced it. 

We draw, also, on our participant observation, which we conducted inside and outside clinical settings. This required IRB approvals and special permissions from the doctors at the clinics, where we observed patient-doctor interactions as well as the staff's workflows. We also accompanied frontline health workers in their house visits to understand the nature of their interactions with the local slum communities.

The data we analyzed was in the form of interview transcripts and field notes, recorded daily over a period of two months. We coded this data iteratively, relying on the memo-writing approach laid out by Charmaz in \cite{}, and highlight findings around access - who had it and who did not, and how it was influenced by the interplay between different actors in the state's healthcare system. We present these findings next. %Can we say more about how Charmaz influenced this process?