\section{Introduction}
\begin{comment}
High level overview of the phenomenon you are studying. Should be something the reader can immediately get behind and hook themselves to. 
Now carry the reader through your flow. Draw their focus to the more narrow context in which you did your work.  
“In this paper, we focus on…” Give the high level description of your research.
“This paper is structured as follows.” Give the outline. When you mention the discussion, also mention the main findings of the discussion.
\end{comment}

Recent HCI research investigates how HCI can be of use to under-served, under-represented, and under-resourced communities across the world. This growing area of research, also known as HCI for Development or HCI4D, includes within its purview of examination the design, deployment, adoption, and use of technologies in the domains of health \cite{PH-CHI;TP}, education \cite{Kam}, agriculture \cite{Patel}, among others. In this paper, we present our findings from a qualitative inquiry into the design and implementation of a public health initiative instituted by the state in New Delhi (India). In a city that is the capital of a country of more than 1 billion and home to over 18 million of its population, in addition to being one of the most densely populated as well, socioeconomic inequalities are high and public resources are strained beyond hope. 

To address the lack of resources in health care, Mohalla (or neighborhood) Clinics were introduced by the Aam Aadmi Party - the governing party in New Delhi - in 2015. Information distributed by the party's officials referred to the Clinics as respite from long lines and inaccessible health care for Delhi's "poorest of the poor" \cite{AAPpressrelease}. Mohalla Clinics are part of a comprehensive three tier health system proposed by the Delhi government to complement the existing network of government hospitals and dispensaries offering free health care  \cite{article}. At the lowest tier of the proposed system are Mohalla clinics  that are operated by one doctor assisted by staff and offer free consultations, medication and over 200 tests. These are explcitly intended to cover the last mile in health care delivery. At the secondary level, polyclinics will be set up that provide specialized services. The existing large government hospitals make up the top-most level of the system. Presently, over a hundred Mohalla Clinics have been set up in Delhi to support the existing network of dispensaries and hospitals which are overwhelmed by the huge population. The government aims to open 1000 such clinics in the city \cite{?}.

In this paper, we present our study of the Mohalla Clinics conducted from May-July 2016. Given the stated goal of these clinics to serve those who cannot currently access quality healthcare as well as the already existing strain on public health resources, we argue that the design of Mohalla Clinics should incorporate factors that encourage use by the underserved population while encouraging non-use by the population with existing access to healthcare. Using this lens of design factors encouraging/discouraging use and non-use, we perform an ethnographic study that aims to shed light on the design decisions made and whether these decisions help realize the design objectives (or not). 

Traditional work on use in HCI focuses on the users and non-users of technology and design recommendations typically focus on factors that aid and enable the target populations to use the technology more effectively. Our study of the Mohalla clinics demonstrates that, for HCI4D interventions in particular, it is equally important to focus on factors that enable both use and non-use by the respective target groups.  We observe that in the absence of such a design philosophy, individuals that fall outside the target group may access these services and those targeted may be excluded by design. Our work extends the understanding of use in the context of interventions such as Mohalla clinics that aim to target specific populations.

\begin{comment}
In this paper, we present our study of the Mohalla Clinics conducted from May-July 2016. Our use of ethnographic methods aims to shed light on the design objectives of this initiative and how these were realized (or not). By discussing the findings from our observations and interviews, we draw attention to the target users of this initiative and the use that resulted from this intervention, highlighting the challenges of (designing such programs).

While traditional work on use in HCI focuses on the users and non-users of technology and design recommendations follow initial assumptions regarding these groups \cite{?}, our study of the Mohalla clinics demonstrates that the distinction between use and non-use can be unclear. Individuals that fall outside the target group may access these services \cite{baumer2015usees} and those targeted may be excluded by design. Our paper draws upon the existing body of literature to extend the understanding of use in the context of interventions such as Mohalla clinics that aim to target specific populations.
\end{comment}