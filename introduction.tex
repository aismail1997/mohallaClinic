\section{Introduction}

A growing focus of recent HCI research has been to investigate how HCI can be of use to underserved communities across the world. Much of this work falls under the umbrella of HCI for Development (HCI4D) and includes within its purview of examination the design, deployment, adoption, and use of technologies in the domains of health \cite{DeRenzi:2008:EIP:1357054.1357174}, education \cite{ho2009human}, agriculture \cite{Patel:2010:AOF:1753326.1753434}, among others. In this note, we present our findings from a qualitative inquiry of the design and implementation of a public health initiative instituted by the state of Delhi (India). In a city that is the capital of a country of more than 1 billion and home to over 18 million of its population \cite{}, in addition to being one of the most densely populated as well, socioeconomic inequalities are high and public resources are strained beyond hope. 

To address the paucity of health care resources in Delhi, the Aam Aadmi Party (the governing party) introduced Mohalla (neighborhood) clinics in 2015 \cite{}. Information disseminated by the party's workers referred to these clinics as respite from long lines and inaccessible healthcare for Delhi's ``weakest sections of society'' \cite{AAPpressrelease}. These clinics form part of a comprehensive three tier health system proposed by the Delhi government to complement the existing network of government hospitals and dispensaries offering free healthcare \cite{article}. At the third and lowest tier, Mohalla clinics cover the last mile in the delivery of healthcare; they are operated by doctors assisted by staff who offer free-of-cost consultations to patients, medications, and over 200 medical tests. Polyclinics and government hospitals make up the second and first tiers, respectively. At present, more than a hundred Mohalla Clinics have been set up across Delhi to support the existing understaffed and overstrained network of dispensaries and hospitals. The Delhi government purportedly aims to open 1,000 such clinics in the city \cite{}.

Given the stated goal of Mohalla Clinics to serve a population that currently has acutely limited access to healthcare, we present research findings that highlight the ways in which these clinics target (or not) the users they have been designed for. For a diversity of factors, which we discuss in this note, targeted users are unable to access these clinics for their healthcare needs, while at the same time, \textit{non-targeted users,} in fact, do avail their services. By examining the case of Mohalla Clinics, we offer recommendations for the design of technology interventions by researchers and practitioners in the global development or the HCI4D space, so that these might be better directed at target groups. We also reflect on the ethical quandaries that this design for use \emph{and non-use} might raise. 

%Our note is structured as follows. We first situate our study in a body of work on use and non-use within the larger HCI community, also relating it to a growing literature in the field of HCI4D. We then describe our methodology, before sharing our analysis of the data we collected for our study. We summarize, finally, our recommendations 

%In this paper, we present our study of the Mohalla Clinics conducted from May-July 2016. Given the stated goal of these clinics to serve those who cannot currently access quality health care as well as the already existing strain on public health resources, we argue that the design of Mohalla Clinics should incorporate factors that encourage [vs mediating] use by the under-served population while encouraging non-use by the population with existing access to health care. Using this lens of design factors encouraging/discouraging use and non-use, we perform an ethnographic study that aims to shed light on the design decisions made and whether these decisions help realize the design objectives (or not). 

%Traditional work on use in HCI focuses on the users and non-users of technology and design recommendations typically focus on factors that aid and enable the target populations to use the technology more effectively.< ADD first monday non-use literature> Our study of the Mohalla clinics demonstrates that, for HCI4D interventions in particular, it is equally important to focus on factors that enable both use and non-use by both the target and non-target groups. %respective target groups - was unclear.  
%We observe that in the absence of such a design philosophy, individuals that fall outside the target group may access these services and those targeted may be excluded by design. Our work extends the understanding of use in the context of interventions such as Mohalla clinics that aim to target specific populations.

% paper structure follows....we talk about

%In this paper, we present our study of the Mohalla Clinics conducted from May-July 2016. Our use of ethnographic methods aims to shed light on the design objectives of this initiative and how these were realized (or not). By discussing the findings from our observations and interviews, we draw attention to the target users of this initiative and the use that resulted from this intervention, highlighting the challenges of (designing such programs).
%While traditional work on use in HCI focuses on the users and non-users of technology and design recommendations follow initial assumptions regarding these groups \cite{?}, our study of the Mohalla clinics demonstrates that the distinction between use and non-use can be unclear. Individuals that fall outside the target group may access these services \cite{baumer2015usees} and those targeted may be excluded by design. Our paper draws upon the existing body of literature to extend the understanding of use in the context of interventions such as Mohalla clinics that aim to target specific populations.