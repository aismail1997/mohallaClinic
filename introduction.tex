\section{Introduction}
\begin{comment}
High level overview of the phenomenon you are studying. Should be something the reader can immediately get behind and hook themselves to. 
Now carry the reader through your flow. Draw their focus to the more narrow context in which you did your work.  
“In this paper, we focus on…” Give the high level description of your research.
“This paper is structured as follows.” Give the outline. When you mention the discussion, also mention the main findings of the discussion.
\end{comment}

\textcolor{red}{}
Recent HCI research investigates how HCI can be of use to under-served, under-represented, and under-resourced communities across the world. This growing area of research, also known as HCI for Development or HCI4D, includes within its purview of examination the design, deployment, adoption, and use of technologies in the domains of health \cite{PH-CHI;TP}, education \cite{Kam}, agriculture \cite{Patel}, among others. In this paper, we present our findings from a qualitative inquiry into the design and implementation of a public health initiative instituted by the state in New Delhi (India). In a city that is the capital of a country of more than 1 billion and home to X (over 18 million) of its population, in addition to being one of the most densely populated as well, socioeconomic inequalities are high and public resources are strained beyond hope. 

(To reduce the strain on resources in the health care sector,) Mohalla (or neighborhood) Clinics were introduced by the Aam Aadmi Party - the governing party in New Delhi - in 2015. Information distributed by the party's officials referred to the Clinics as respite from long lines and inaccessible health care for Delhi's "poorest of the poor" (replace with "weakest sections?") \cite{AAPpressrelease}. [Quick background.]

To this effect, over 100 Mohalla Clinics have been opened in Delhi to supplement, and perhaps even replace, the existing network of dispensaries and hospitals which are overwhelmed by the huge population. The government aims to open 1000 such clinics in the city \cite{?}.

In this paper, we present our study of the Mohalla Clinics conducted from May-July 2016. Our use of ethnographic methods aims to shed light on the design objectives of this initiative and how these were realized (or not). By discussing the findings from our observations and interviews, we draw attention to the target users of this initiative and the use that resulted from this intervention, highlighting the challenges of (designing such interventions).

While traditional HCI work focuses on the users and non-users of technology and design recommendations follow initial assumptions regarding these groups \cite{?}, our study of the Mohalla clinics demonstrates that the distinction between use and non-use can be unclear. Individuals that fall outside the target group may access these services \cite{Baumer} and those targeted may be excluded by design. Our paper draws upon the existing body of literature to extend the understanding of use in the context of interventions such as Mohalla clinics that aim to target specific populations.
\begin{comment}

Case study of mohalla clinics - who gets benefits, who doesn’t? Who are the stakeholders - diff groups? Targeting the poorest of the poor - articles: toi, etc

*****
2: While designing - how do we look at use? And whether the use is exclusive - possible? 
Geographical locations: high socio-economic gap - same geographical location - harder to segregate - urban - How do we ensure services are accessed by poor

3: Case study of clinics - overview
We present a case study of two neighborhood clinics or 'Mohalla clinics' set up by the government in the same geographical location. 

4: What is the paper about?
In this paper, we examine - and the ethical(?) implications of designing for a particular set of users.
\end{comment}