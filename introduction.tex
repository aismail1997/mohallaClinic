\section{Introduction}

\begin{comment}
High level overview of the phenomenon you are studying. Should be something the reader can immediately get behind and hook themselves to. 
Now carry the reader through your flow. Draw their focus to the more narrow context in which you did your work.  
“In this paper, we focus on…” Give the high level description of your research.
“This paper is structured as follows.” Give the outline. When you mention the discussion, also mention the main findings of the discussion.
\end{comment}

\textcolor{red}{}

Recent HCI research investigates how HCI can be of use to underserved, under-represented, and under-resourced communities across the world. This growing area of research, also known as HCI for Development or HCI4D, includes within its purview of examination the design, deployment, adoption, and use of technologies in the domains of health \cite{PH-CHI;TP}, education \cite{Kam}, agriculture \cite{Patel}, among others. In this paper, we present our findings from a qualitative inquiry into the design and implementation of a public health initiative instituted by the state in New Delhi (India). In a city that is the capital of a country of more than 1 billion and home to X\% of its population, in addition to being the most densely populated as well, socioeconomic inequalities are high and public resources are strained beyond hope. 

Mohalla Clinics were introduced by the Aam Aadmi Party - the governing party in New Delhi - in 2015. Information distributed by the party's officials referred to the Clinics as respite from long lines and inaccessible health care for Delhi's ``poorest of the poor'' \cite{}. [Quick background.]

In this paper, we present our study of the Mohalla Clinics conducted from May-July 2016. Our use of ethnographic methods aims to shed light on the design objectives of this initiative and how these were realized (or not). By discussing the findings from our observations and interviews, we draw attention to the target users of this initiative and the use that resulted from this intervention, highlighting the challenges of 

1:ICTD - Field of HCI designing for under served underrepresented - cite papers this is the kind of work happening - what research looks like - Q? - who are you designing for?
****

This paper contributes to a growing body of literature in ICTD on designing for under-served and underrepresented communities. []

2: While designing - how do we look at use? And whether the use is exclusive - possible? 
Geographical locations: high socio-economic gap - same geographical location - harder to segregate - urban - How do we ensure services are accessed by poor
*****

While much work has been done studying use and non-use[], less work has been done on understanding whether the service is being used by intended users of the system (add info: usees - Baumer - talks about it positively as we now have more users of the service, but what about the opp case? can it hinder those intended to use the service) and the implications of this. In urban areas such as Delhi where high socioeconomic differences are found in the same geographical location and the same services are accessed by all, this question becomes highly relevant.

3: Case study of clinics - overview
We present a case study of two neighborhood clinics or 'Mohalla clinics' set up by the government in the same geographical location. 

4: What is the paper about?
In this paper, we examine - and the ethical(?) implications of designing for a particular set of users.
