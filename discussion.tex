\section{Discussion}

\textcolor{red}{[To be completed]}

Our findings demonstrate the background processes involved in use and non-use of the Mohalla clinics by target and non-target groups. Since these processes determine the use of the clinic and consequently its impact, it is necessary to pay attention to them during the design process and specifically design the intervention for use by the identified target group (and perhaps non-use by those outside the target group). In the case of Mohalla clinics and health care interventions in general, greater attention needs to be paid to the timings, location, role of the actors, and underlying socio-cultural processes involved in the use of health care to ensure that it truly reaches those in 'need' of the intervention.

To this effect, we identify the scope for ICTD and make design recommendations for Mohalla clinics. While having additional hours in the evening and setting up clinics in areas that do not currently have access to affordable care may ensure use by the target population, these are outside our scope(?) as researchers. Instead we suggest measures to supplement the network of health care providers to redirect the flow of resources towards areas of 'need'. ICTD can support the work of ASHAs, ANMs, Mohalla clinics and dispensaries and leverage informal systems to (spread awareness, care) (how do we extend on previous ICTD work of this sort.)

Current use of technology at the Mohalla clinics is limited to using a tablet to record the patient's identifying information and medicines provided. (instead collect data that is useful and can help improve the intervention and identify places for improvement?)

% ethically designing for use and non-use 



%* discuss our classification further - while we have used this classification, people falling in between - challenges we faced with this classification - Perspective of people who are coming, not coming, doctors - cannot say that they are not coming, cannot conclusively cannot say that middle income are coming. There’s a gap - how can you close the gap further - how to make it more effective?

%* Improve design of the Mohalla clinics (includes timings and location) to make it more accessible- make design recommendations - improving function of dispensaries over building more clinics?

%* awareness of the clinics - announcement systems for immunization etc - leverage such informal systems here?

%* Scope for ICTD - supplement network of hospitals, health care providers, patients - current use of tech very limited, design recommendations - info dissemination - support systems

%* Are design objectives met? collecting data that is "useful" and provide a feedback system to improve functioning and design of the clinics

%* category - discouraging use by non-target - is it ethical to actively do that? perhaps focusing on where there is a need vs actively discouraging others

%Class + socio-economic: look at references for class

%Designing services and challenges involved, how to ensure services are adopted?

%Design recommendations -
%Waiting time - ICTD paper 
%Info factor, how it is provided eg morning, location
%Sustainability of the intervention

%Factors leading to what is overlooked and say stuff was overlooked
%Extending the service to other 

%Is there a motivation for them to go to a clinic 

%Other aspect: how do you share info about the intervention - info gap, social gap
%Incentives to stay engaged