\section{Discussion}

\begin{comment}
Depends on what the nature of your study was, but you might either leave the reader with design recommendations for the context you studied (e.g., how could HCI address the lives of young women in rural West Bengal facing constant threats to their safety) or else discuss the main takeaways for your desired readers of this paper (ICTD researchers? Other HCI researchers? Practitioners?). This needs thought and work. You need a week just to think about and iterate on the Discussion. Do not take this lightly. If you are taking on a particular theoretical lens, you will need to deeply engage with this lens through the Discussion. 
\end{comment}

\textcolor{red}{}
\begin{comment}
How ethics play a role? How to still reach target

Class + socio-economic: look at references for class

Designing services and challenges involved, how to ensure services are adopted?

Design recommendations -
Waiting time - ICTD paper 
Info factor, how it is provided eg morning, location
Sustainability of the intervention

Factors leading to what is overlooked and say stuff was overlooked
Extending the service to other 

What were the Design objectives?
How can we check if they are met
Collecting the right data
 
Time factor - why is it that time when people are working
Perspective of ppl who are coming, not coming, drs - cannot say that they are not coming, cannot conclusively cannot say that middle income are coming. There’s a gap - how can you close the gap further - how to make it more effective?

Is there a motivation for them to go to a clinic 

Collecting better data to assess interventions - data that is easy to collect 
Other aspect: how do you share info about the intervention - info gap, social gap
Footfall track to monitor for number of ppl coming to the clinic
Incentives to stay engaged
\end{comment}

