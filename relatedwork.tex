\section{Related Work}
%Intro

%\subsection{HCI work - ICTD - health care}
%marginalized vs not
%rural generally
%easy to carve generally - example rural - phc specific audience

%\subsection{Use and non-use - designing for communities}
% ethically conflicted stance - deciding who to target
In their discussion on research regarding use and non-use, Baumer et al. \cite{Baumer:2015:USE:2702123.2702147} mention that there are many ways of conceptualizing use and non-use, also adding that there are shades in between – as also discussed by Lenhart and Horrigan \cite{2003} – that we might sometimes miss. Baumer et al. \cite{2015} further discuss the ``digital divide'' \cite{Keniston2003} as a factor that induces non-use of technologies - when an individual might wish to use a technology but cannot, because those technologies may not be accessible, affordable, or available. We see parallels in the case of the Mohalla Clinics, where individuals and communities end up being excluded from making use of these clinics. No one is actively excluding them, however - no individuals, communities, or systems. In fact, it is not even the costs that are preventing use, as in the case of Wyche et al.'s work on Facebook use in rural Kenya \cite{2013}. There are several contributing factors, however, including some semblance of caste/class/gender hierarchies that kept out target users, \textit{per se}, as in \cite{Patra2007}, who studied computer sharing in Indian schools and observed that children from lower castes (or girls) were left out of the fray. In Burrell's work \cite{2012}, we find a parallel flavor of exclusion, given its study of young men who frequented Internet cafes in Ghana hoping to meet and communicate with foreigners online; though these men were `connected’, they were subject to an online culture that followed Euro-American norms as well as low quality and expensive access to connectivity. In our research, we discovered a similar trend, where we found that even though targeted non-users (or `false negatives') were permitted in these Clinics and non-targeted users (or `false positives') were not expressly sought, external, sociocultural factors determined use. 