\section{Related Work}
\begin{comment}
There are two, or three ways to do this. The first is to not do a RW section at all. In this case, you let the data/analysis connect with prior work as you go along. It would be nice if this could be sufficient but most often it is not. Further, people expect a section on prior work so you make it easier for reviewers to understand which body of work you are drawing on and extending.
 And yes, always draw on and extend. You want to contribute to a body of knowledge that already exists, because together as academics, that is our goal. However, you also want to extend that body of knowledge (otherwise why write this paper?). 
The second way to do this is to break RW down into subsections that are like circles in a venn diagram. Your research falls into the intersection of those circles. They may or may not intersect at multiple points. The important thing is to cover every paper that has a leaning/focus similar to yours. So you could do -- “Our paper builds on XYZ work. Here’s X and here’s how we extend it. Here’s Y and …”
The third (and slightly more preferable way, according to me) is to make the related work section flow instead of have it broken down into these discrete sections like I just mentioned. Tell it like a story, not like a list of papers. Make it seem thoughtful and nuanced, instead of a “1, 2, 3” listing. 
You can look at the Mobile Phones for Maternal Health in Rural India paper that I wrote for CHI 2015 as an example. It is by no means the best way to do this, but it did not raise any eyebrows in the review process, at least. One subsection (or 2-3 paragraphs) could be dedicated just to the theoretical lens, if you are using one.
\end{comment}
\textcolor{red}{}
\subsection{HCI work - ICTD}
\subsection{ICTD work in health care}
\subsection{Use and designing for communities}

