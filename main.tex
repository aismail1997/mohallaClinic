\documentclass{sigchi}
\pagenumbering{arabic}


% Load basic packages
\usepackage{balance}  % to better equalize the last page
\usepackage{graphics} % for EPS, load graphicx instead
\usepackage{times}    % comment if you want LaTeX's default font
\usepackage{url}      % llt: nicely formatted URLs

\makeatletter
\def\url@leostyle{
  \@ifundefined{selectfont}{\def\UrlFont{\sf}}{\def\UrlFont{\small\bf\ttfamily}}}
\makeatother
\urlstyle{leo}

\def\pprw{8.5in}
\def\pprh{11in}
\special{papersize=\pprw,\pprh}
\setlength{\paperwidth}{\pprw}
\setlength{\paperheight}{\pprh}
\setlength{\pdfpagewidth}{\pprw}
\setlength{\pdfpageheight}{\pprh}

\usepackage[pdftex]{hyperref}
\hypersetup{
pdftitle={Designing Neighborhood Clinics for...},
pdfauthor={Azra Ismail},
pdfkeywords={SIGCHI, proceedings},
bookmarksnumbered,
pdfstartview={FitH},
colorlinks,
citecolor=black,
filecolor=black,
linkcolor=black,
urlcolor=black,
breaklinks=true,
}

\newcommand\tabhead[1]{\small\textbf{#1}}

\begin{document}

%this may change
\title{Designing Neighborhood Clinics towards Targeted Use}

\numberofauthors{3}
\author{
  \alignauthor 1st Author Name\\
    \affaddr{Affiliation}\\
    \affaddr{Address}\\
    \email{e-mail address}\\
    \affaddr{Optional phone number}
  \alignauthor 2nd Author Name\\
    \affaddr{Affiliation}\\
    \affaddr{Address}\\
    \email{e-mail address}\\
    \affaddr{Optional phone number}    
  \alignauthor 3rd Author Name\\
    \affaddr{Affiliation}\\
    \affaddr{Address}\\
    \email{e-mail address}\\
    \affaddr{Optional phone number}
}

\maketitle

\begin{abstract}
\begin{comment}
You should write an abstract early on, knowing fully well that you will (and should) iterate through it at least 3-4 times. It is also the thing you should edit again just before you submit.

Example:
"We present the design, deployment, and mixed-methods evaluation of an interactive education software developed to circumvent taboos in teaching youth in India about HIV prevention. While extremely important, providing complete and accurate HIV education in India has historically been difficult because of cultural taboos around the virus'€™ method of transmission. To address this gap, we used an iterative design process to develop a culturally appropriate educational software that leverages the unique affordances of technology and key theories from the learning sciences and communication. We demonstrate that this software leaves students significantly more knowledgeable about HIV transmission and prevention and with reduced stigma toward individuals infected with HIV. Validating its effectiveness in circumventing taboos, the software has since been adopted by numerous local governments in India where other HIV education materials had been banned, and students report extreme comfort learning from it. The methodology presented here has broader implications for the development and implementation of technology designed to educate on sensitive topics in health and other areas."
\end{comment}
\textcolor{red}{[We present a study/examination/analysis of - We do X to understand Y. We highlight X, discussing Y, and conclude with Z.]
1. What this paper is about
2. Motivation
3. Approach
4. Findings
5. Broader applicability
6. Paper takeaways
}
\end{abstract}

\keywords{ICTD; technology; India}

\category{H.5.m.}{Information Interfaces and Presentation
  (e.g. HCI)}{Miscellaneous} 

\section{Introduction}
\begin{comment}
High level overview of the phenomenon you are studying. Should be something the reader can immediately get behind and hook themselves to. 
Now carry the reader through your flow. Draw their focus to the more narrow context in which you did your work.  
“In this paper, we focus on…” Give the high level description of your research.
“This paper is structured as follows.” Give the outline. When you mention the discussion, also mention the main findings of the discussion.
\end{comment}

Recent HCI research investigates how HCI can be of use to under-served, under-represented, and under-resourced communities across the world. This growing area of research, also known as HCI for Development or HCI4D, includes within its purview of examination the design, deployment, adoption, and use of technologies in the domains of health \cite{PH-CHI;TP}, education \cite{Kam}, agriculture \cite{Patel}, among others. In this paper, we present our findings from a qualitative inquiry into the design and implementation of a public health initiative instituted by the state in New Delhi (India). In a city that is the capital of a country of more than 1 billion and home to over 18 million of its population, in addition to being one of the most densely populated as well, socioeconomic inequalities are high and public resources are strained beyond hope. 

To address the lack of resources in health care, Mohalla (or neighborhood) Clinics were introduced by the Aam Aadmi Party - the governing party in New Delhi - in 2015. Information distributed by the party's officials referred to the Clinics as respite from long lines and inaccessible health care for Delhi's "poorest of the poor" \cite{AAPpressrelease}. These are part of a comprehensive three tier health system proposed by the Delhi government to complement the existing network of government hospitals and dispensaries offering free health care \cite{article}. At the lowest tier of the proposed system are Mohalla clinics that are operated by one doctor assisted by staff and offer free consultations, medication and over 200 tests. These are explicitly intended to cover the last mile in health care delivery. At the secondary level, polyclinics will be set up that provide specialized services. The existing large government hospitals make up the top-most level of the system. Presently, over a hundred Mohalla Clinics have been set up in Delhi to support the existing network of dispensaries and hospitals which are overwhelmed by the huge population. The government aims to open 1000 such clinics in the city \cite{?}.

In this paper, we present our study of the Mohalla Clinics conducted from May-July 2016. Given the stated goal of these clinics to serve those who cannot currently access quality health care as well as the already existing strain on public health resources, we argue that the design of Mohalla Clinics should incorporate factors that encourage use by the under-served population while encouraging non-use by the population with existing access to health care. Using this lens of design factors encouraging/discouraging use and non-use, we perform an ethnographic study that aims to shed light on the design decisions made and whether these decisions help realize the design objectives (or not). 

Traditional work on use in HCI focuses on the users and non-users of technology and design recommendations typically focus on factors that aid and enable the target populations to use the technology more effectively. Our study of the Mohalla clinics demonstrates that, for HCI4D interventions in particular, it is equally important to focus on factors that enable both use and non-use by both the target and non-target groups. %respective target groups - was unclear.  
We observe that in the absence of such a design philosophy, individuals that fall outside the target group may access these services and those targeted may be excluded by design. Our work extends the understanding of use in the context of interventions such as Mohalla clinics that aim to target specific populations.
\begin{comment} 
In this paper, we present our study of the Mohalla Clinics conducted from May-July 2016. Our use of ethnographic methods aims to shed light on the design objectives of this initiative and how these were realized (or not). By discussing the findings from our observations and interviews, we draw attention to the target users of this initiative and the use that resulted from this intervention, highlighting the challenges of (designing such programs).
While traditional work on use in HCI focuses on the users and non-users of technology and design recommendations follow initial assumptions regarding these groups \cite{?}, our study of the Mohalla clinics demonstrates that the distinction between use and non-use can be unclear. Individuals that fall outside the target group may access these services \cite{baumer2015usees} and those targeted may be excluded by design. Our paper draws upon the existing body of literature to extend the understanding of use in the context of interventions such as Mohalla clinics that aim to target specific populations.
\end{comment}
\section{Background}

\begin{comment}
Your paper may or may not need this section, but if you think that it might benefit from it, err on the side of caution. Write about the story behind your research. Build up the context that the reader could read, like a story, to understand more about the problem you’re addressing, the phenomena you are talking about. If the paper is about an internet ban in Bangladesh, tell the story about how the ban was put in place, what was the timeline like, who said what in popular press, and so on. If the research was conducted in the context of an organization of health/outreach workers, talk about the organization, its charter, what it has done/is doing. If the paper is about a particular tech practice, such as emoji communication amongst Chinese users, talk about the history of mobile communication in that context (in addition to what you will say in the Related Work section). If the paper is about a tool that was deployed and evaluated, write about the tool and elaborate on its features.
\end{comment}

\textcolor{red}{}

[edit this: taken from post]

With its massive population of over a billion people, India is faced with the logistical challenge of dispensing quality health care to each and every citizen irrespective of geographical location and economic status. Today, even in the capital city of Delhi, there is a vast difference in the level of access and quality of health care received by people from different socio-economic groups.

The Delhi government has proposed an interesting way to tackle this. It is planning to set up a 3-tier system where the lowest tier is a system of ‘Mohalla Clinics’, government clinics in each neighborhood which provide free consultation, tests and medicines. While the existing dispensaries are meant to cater to 50,000 people, each clinic will cater to around 10,000 people in a locality. So far around a hundred clinics have been set up and the government aims to open 1000 such clinics by the end of the year.

According to one doctor I met, the clinics have been set up haphazardly by the government as an immediate response to the crisis. Mohalla clinics are being set up in rented rooms or with pre-made structures around the city. Those running them are doctors coming out of retirement or wanting to give back to society. The government pays the doctors around Rs 30 for every patient they see which is lower than what they would have received at a private practice. The clinics in operation have little in common save the promise to provide quality health care, the facilities available and the staff employed vary.

The two clinics in my area were open only from 9am to 1pm, 6 days a week, during the time many people work. One clinic I visited had a waiting time of around 2 hours while the other had practically no waiting time, though both saw the same number of visitors in a day. At the moment, the clinics are relying on word of mouth to bring new customers. 

*****

To [high level aim of the initiative, cite here, press release?], the Delhi government has proposed a 3-tier system that supports the existing network of government hospitals and dispensaries that offer free health care. The goal is [].

At the lowest level of this system are neighborhood clinics or 'Mohalla clinics', each of which cater to 10,000 people living in that locality. Each of these clinics is manned by one doctor and provides free consulting, free medicines and tests. These clinics operate from 9am to 1pm, six days a week.[]

The next tier of the proposed system involves polyclinics that offer specialized services [].

...

As of August 2016, there are 108[check] Mohalla clinics and 2[check] polyclinics set up by the government. Our study will be focusing on two Mohalla clinics. 
\section{Related Work}
\textcolor{red}{[to be completed]} 
\subsection{HCI work - ICTD}
\subsection{ICTD work in health care}
\subsection{Use and designing for communities}

\section{Methodology}

\begin{comment}
This should tell the story of what you did, and the single most important point is to show that you were reflective, rigorous, ethical, sound in your research. 
Start with where you conducted the study. Was there one site or many? How long was your (the authors’) engagement with this site? What was the nature of the engagement? And remember to anonymize by default.
Say what methods you used and who you studied. How long were your interviews? What kinds of questions did you ask? Did you use an interpreter? Who were these people? How old?
Add a paragraph on who the authors are and what they bring to the table. Say what their biases might have been that could be limitations of this work. Was there access you could not get? 
How did you analyze the data? If you used grounded theory, whose version of grounded theory did you use? Who was responsible for the analysis? 
Finally, there are lots of papers that have a decent write-up for methodology. Find a researcher you like and read up on the Methodology section.
\end{comment}
%\textcolor{red}{[Make this flow. Content is here]}

The multi-site short-term ethnographic study was conducted in urban Delhi from May to July 2016, through a process of semi-structured interviews and observations. The data collected was in the form of audio recordings, field notes and interview transcripts. Interview questions revolved around understanding the nature of the interactions with [workers vs actors] (frontline + experts) in the health care system, engagement with the facilities available and perception of health and health care.

Our study was limited to two sites around six kilometers apart. One of these locations was a slum under a flyover at the intersection of major roads and the researcher's engagement with this site lasted a week. The other site was a market area of a radius of about 500 meters and the research was conducted in multiple settings in this area. 
The researcher conducting the interviews and observations is intimately familiar with this area and we are aware of the preconceived notions and biases this brings to the table, albeit with a deeper understanding of this region.
% all of the data collected by first author - spoke fluently with them in native language  
The population studied was comprised primarily of migrants from Bihar, Uttar Pradesh and Bengal 
%(mostly Muslim?)
and were between 18 to 68 years old. The researchers communicated with the subjects in the colloquial language which is a mix of Hindi, Urdu, and English.

The settings studied included clinical settings - 3 private clinics, 2 Mohalla Clinics, a government dispensary and non-clinical settings - slums, in an effort to paint a more complete picture of the health care system in Delhi. During the course of the study, we interacted with 5 doctors, 4 staff members, 5 front-line health workers (4 ASHAs and 1 ANM) and 50 potential users. Interviews in clinical settings typically lasted between 15-30 minutes though some interactions lasted over an hour. Interviews conducted outside the clinical setting lasted around an hour. Engagement with the front-line health workers and potential users also involved participant observation where we accompanied 2 ASHAs and an ANM during their field work in slums along the Yamuna river for three days.

%coding? - how did we do analysis?
We used Charmaz's grounded theory 
% inductive analysis - x aspect of Charmaz's approach - write memos  
to analyze our data. Data collected was analyzed by the researcher conducting the interviews and observations. [add stuff about coding and analysis - ]
% !!!!!!!
Despite the attempt to paint a complete picture, an important missing piece of our study is the top-down perspective. We were unable to gain the government's perspective during this study. Our knowledge about their goals, ideas and future plans is obtained from press releases, news articles and popular media. 

%is it - change this based on sources we actually use
\section{Findings}
\begin{comment}
This will form the chunk of your writing. I have sometimes iterated through these in written form, or else made a rough outline  on paper and then put it down. It is best that we talk about this section before you attempt to write. Depending on the structure of this section, you might label the section Findings or Analysis. The thing to remember is that all of your data goes into this section, and you need to weave a story around your data that is compelling, novel, and forms the central theme of the paper. If you have done qualitative work, make sure you include these quotes in the findings. Most importantly, do not tell a story based on quotes you have. Pick quotes based on the story you want to tell.

STORY OF USE
\end{comment}
\textcolor{red}{We present our findings from our field work.... [Rough categories]}

\subsection{Challenges and Affordances (of accessing)}
sustainability - relying on goodwill - low pay + renting premises

\subsection{Design of the intervention}
Design - of the tablet/location/hours, could tie back to the challenges
timing: (open 9 - 1)
tablet: limited use - only recording the  (see pics of app)
location: 2 clinics - same geographical location - 10 minute walk from each other - one near the yamuna river where many of the slums are located. other flats

\subsection{Actors}
Actors - 3 layers - front-line health workers, doctors, patients (also seems to have a strong element of care)

Quotes and observations that may be relevant and/or interesting:
1. Look for convo with maid: she talks about when she can go and her surprise and being unaware of the location of the clinic (design)
2. pick one of the quotes on "live right here" (design)
3. quote by ASHAs on "forcing people" and "threatening them with consequences of refusing immunization" (actors)
4. religion - as a barrier - BUT - used by ASHAs as a tool to connect with ppl and further their outreach (actors)
5. gender - barrier/facilitator in access? - how gender roles = improved/worse access to health? (affordances + actors?)
6. need for empathy - abuse - (actors) 
7. Quote on low pay (challenges?)
8. Patients come with multiple problems (affordances?)
9. Impact of MC (affordances)
10. Perceived care, Perception of MC (affordances + actors?)
11. Priority of dr, Expectations from doctors (actors)
12. Expectations of free, Expectations of MC (affordances?)
13. 



\section{Discussion}

\textcolor{red}{[To be completed]}

\begin{comment}
Depends on what the nature of your study was, but you might either leave the reader with design recommendations for the context you studied (e.g., how could HCI address the lives of young women in rural West Bengal facing constant threats to their safety) or else discuss the main takeaways for your desired readers of this paper (ICTD researchers? Other HCI researchers? Practitioners?). This needs thought and work. You need a week just to think about and iterate on the Discussion. Do not take this lightly. If you are taking on a particular theoretical lens, you will need to deeply engage with this lens through the Discussion. 
\end{comment}

\begin{comment}
How ethics play a role? How to still reach target

Class + socio-economic: look at references for class

Designing services and challenges involved, how to ensure services are adopted?

Design recommendations -
Waiting time - ICTD paper 
Info factor, how it is provided eg morning, location
Sustainability of the intervention

Factors leading to what is overlooked and say stuff was overlooked
Extending the service to other 

What were the Design objectives?
How can we check if they are met
Collecting the right data
 
Time factor - why is it that time when people are working
Perspective of ppl who are coming, not coming, drs - cannot say that they are not coming, cannot conclusively cannot say that middle income are coming. There’s a gap - how can you close the gap further - how to make it more effective?

Is there a motivation for them to go to a clinic 

Collecting better data to assess interventions - data that is easy to collect 
Other aspect: how do you share info about the intervention - info gap, social gap
Footfall track to monitor for number of ppl coming to the clinic
Incentives to stay engaged
\end{comment}



\section{Conclusion}
%\textcolor{red}{Say what you did. Say what you (or one) might do in the future. If you don’t want to do the latter, don’t, and then don’t have the “Future Work” in there. }

\textcolor{red}{[To be completed]}
\balance

\bibliographystyle{sigchi}
\bibliography{references}

\end{document}
