\documentclass{sigchi}
\pagenumbering{arabic}


% Load basic packages
\usepackage{balance}  % to better equalize the last page
\usepackage{graphics} % for EPS, load graphicx instead
\usepackage{times}    % comment if you want LaTeX's default font
\usepackage{url}      % llt: nicely formatted URLs

\makeatletter
\def\url@leostyle{
  \@ifundefined{selectfont}{\def\UrlFont{\sf}}{\def\UrlFont{\small\bf\ttfamily}}}
\makeatother
\urlstyle{leo}

\def\pprw{8.5in}
\def\pprh{11in}
\special{papersize=\pprw,\pprh}
\setlength{\paperwidth}{\pprw}
\setlength{\paperheight}{\pprh}
\setlength{\pdfpagewidth}{\pprw}
\setlength{\pdfpageheight}{\pprh}

\usepackage[pdftex]{hyperref}
\hypersetup{
pdftitle={Designing for Use and Non-Use: A Perspective from Neighborhood Clinics},
pdfauthor={Azra Ismail},
% * <aismail1997@gmail.com> 2016-09-13T12:25:10.966Z:
%
% ^.
pdfkeywords={SIGCHI, proceedings},
bookmarksnumbered,
pdfstartview={FitH},
colorlinks,
citecolor=black,
filecolor=black,
linkcolor=black,
urlcolor=black,
breaklinks=true,
}

\newcommand\tabhead[1]{\small\textbf{#1}}

\begin{document}

%change this!!! - not sure what we mean
\title{Designing for Use and Non-Use: A Perspective from Neighborhood Clinics}
\numberofauthors{3}
\author{
  \alignauthor 1st Author Name\\
    \affaddr{Affiliation}\\
    \affaddr{Address}\\
    \email{e-mail address}\\
    \affaddr{Optional phone number}
  \alignauthor 2nd Author Name\\
    \affaddr{Affiliation}\\
    \affaddr{Address}\\
    \email{e-mail address}\\
    \affaddr{Optional phone number}    
  \alignauthor 3rd Author Name\\
    \affaddr{Affiliation}\\
    \affaddr{Address}\\
    \email{e-mail address}\\
    \affaddr{Optional phone number}
}

\maketitle
\begin{abstract}
We present a qualitative study of Mohalla (neighborhood) Clinics - an intervention introduced by the state government of Delhi (India) in 2015 to provide access to basic health care. Though these clinics were introduced with the explicit objective of targeting the ``weakest sections of society,'' their actual use tells a different story. We conducted interviews and participant observations to study two Mohalla Clinics and their use by local residents who belonged to diverse socioeconomic groups. Our findings highlight the factors influencing the use of the clinics by targeted underserved sections of society when the same services are also desired (and availed) by non-targeted groups. We draw on these findings to inform the design of interventions, particularly in the context of use and non-use by identified target groups.

\end{abstract}

\keywords{Public health; India; use; non-use; HCI4D}

\category{H.5.m.}{Information Interfaces and Presentation
  (e.g. HCI)}{Miscellaneous} 

\section{Introduction}
\begin{comment}
High level overview of the phenomenon you are studying. Should be something the reader can immediately get behind and hook themselves to. 
Now carry the reader through your flow. Draw their focus to the more narrow context in which you did your work.  
“In this paper, we focus on…” Give the high level description of your research.
“This paper is structured as follows.” Give the outline. When you mention the discussion, also mention the main findings of the discussion.
\end{comment}

Recent HCI research investigates how HCI can be of use to under-served, under-represented, and under-resourced communities across the world. This growing area of research, also known as HCI for Development or HCI4D, includes within its purview of examination the design, deployment, adoption, and use of technologies in the domains of health \cite{PH-CHI;TP}, education \cite{Kam}, agriculture \cite{Patel}, among others. In this paper, we present our findings from a qualitative inquiry into the design and implementation of a public health initiative instituted by the state in New Delhi (India). In a city that is the capital of a country of more than 1 billion and home to over 18 million of its population, in addition to being one of the most densely populated as well, socioeconomic inequalities are high and public resources are strained beyond hope. 

To address the lack of resources in health care, Mohalla (or neighborhood) Clinics were introduced by the Aam Aadmi Party - the governing party in New Delhi - in 2015. Information distributed by the party's officials referred to the Clinics as respite from long lines and inaccessible health care for Delhi's "poorest of the poor" \cite{AAPpressrelease}. Mohalla Clinics are part of a comprehensive three tier health system proposed by the Delhi government to complement the existing network of government hospitals and dispensaries offering free health care  \cite{article}. At the lowest tier of the proposed system are Mohalla clinics  that are operated by one doctor assisted by staff and offer free consultations, medication and over 200 tests. These are explcitly intended to cover the last mile in health care delivery. At the secondary level, polyclinics will be set up that provide specialized services. The existing large government hospitals make up the top-most level of the system. Presently, over a hundred Mohalla Clinics have been set up in Delhi to support the existing network of dispensaries and hospitals which are overwhelmed by the huge population. The government aims to open 1000 such clinics in the city \cite{?}.

In this paper, we present our study of the Mohalla Clinics conducted from May-July 2016. Given the stated goal of these clinics to serve those who cannot currently access quality healthcare as well as the already existing strain on public health resources, we argue that the design of Mohalla Clinics should incorporate factors that encourage use by the underserved population while encouraging non-use by the population with existing access to healthcare. Using this lens of design factors encouraging/discouraging use and non-use, we perform an ethnographic study that aims to shed light on the design decisions made and whether these decisions help realize the design objectives (or not). 

Traditional work on use in HCI focuses on the users and non-users of technology and design recommendations typically focus on factors that aid and enable the target populations to use the technology more effectively. Our study of the Mohalla clinics demonstrates that, for HCI4D interventions in particular, it is equally important to focus on factors that enable both use and non-use by the respective target groups.  We observe that in the absence of such a design philosophy, individuals that fall outside the target group may access these services and those targeted may be excluded by design. Our work extends the understanding of use in the context of interventions such as Mohalla clinics that aim to target specific populations.

\begin{comment}
In this paper, we present our study of the Mohalla Clinics conducted from May-July 2016. Our use of ethnographic methods aims to shed light on the design objectives of this initiative and how these were realized (or not). By discussing the findings from our observations and interviews, we draw attention to the target users of this initiative and the use that resulted from this intervention, highlighting the challenges of (designing such programs).

While traditional work on use in HCI focuses on the users and non-users of technology and design recommendations follow initial assumptions regarding these groups \cite{?}, our study of the Mohalla clinics demonstrates that the distinction between use and non-use can be unclear. Individuals that fall outside the target group may access these services \cite{baumer2015usees} and those targeted may be excluded by design. Our paper draws upon the existing body of literature to extend the understanding of use in the context of interventions such as Mohalla clinics that aim to target specific populations.
\end{comment}
\section{Related Work}
\textcolor{red}{[to be completed]} 
\subsection{HCI work - ICTD}
\subsection{ICTD work in health care}
\subsection{Use and designing for communities}

\section{Methodology}

\begin{comment}
This should tell the story of what you did, and the single most important point is to show that you were reflective, rigorous, ethical, sound in your research. 
Start with where you conducted the study. Was there one site or many? How long was your (the authors’) engagement with this site? What was the nature of the engagement? And remember to anonymize by default.
Say what methods you used and who you studied. How long were your interviews? What kinds of questions did you ask? Did you use an interpreter? Who were these people? How old?
Add a paragraph on who the authors are and what they bring to the table. Say what their biases might have been that could be limitations of this work. Was there access you could not get? 
How did you analyze the data? If you used grounded theory, whose version of grounded theory did you use? Who was responsible for the analysis? 
Finally, there are lots of papers that have a decent write-up for methodology. Find a researcher you like and read up on the Methodology section.
\end{comment}

\textcolor{red}{}

The study was conducted in urban(?) Delhi following a qualitative approach using interviews and observations. The study was limited to a region of Delhi (should mention migrants etc?) and was conducted in English, Hindi and Urdu.

We interacted with doctors, staff and patients in private and government clinics to obtain their perspective in the process of accessing health care. Each interview lasted around 30 minutes with those patients often lasting longer. 

Outside the clinical settings, we interacted with ASHAs and potential users of the health care system in an attempt to build a more complete picture of health care access in this region. These ASHAs worked in slums or 'jhuggis' and we accompanied the ASHAs during their field visits. The interviews in these settings often lasted over an hour and were a minimum of 30 minutes.

****
Settings:
Government dispensary (MCD)
2 Mohalla clinics
Places visited with ASHAs
slums near the Yamuna
Flats
Pvt - 2 small clinics and a hospital (Al - Shifa)
Slum under a flyover

Interviews with doctors, staff, patients and observations in clinics + accompanied ASHA workers
Audio + interview transcripts + field notes
Ppl: migrants from UP, Bihar, Bengal
Hindi/Urdu/English
\section{Findings}
\textcolor{red}{}
%In this section, we discuss our findings about the design and implementation of Mohalla clinics. In particular, we focus on factors that affect the use of these clinics by the target population and the non-use of these clinics by the non-target population. We consider, both, factors that encourage and discourage the desired behaviors. Thus, we classify our findings into four different categories. Factors that: (i) encourage use by target population, (ii) discourage use by target population, (iii) encourage non-use by non-target population, (iv) discourage non-use by non-target population.
%(RM: Assuming target population is folks from lower socio-economic background and non-target population is everyone else i.e. folks from higher socio-economic background.  Will have mentioned this in the intro.) 
Our analysis of the data collected revealed a story of use around Mohalla clinics and other government health care services. Other themes of gender, religion, education, socioeconomic and cultural factors emerged from our data, these revolve around the larger theme of use and non-use. To discuss these themes in the context of use, we classify our findings into four different categories - factors that: (i) encourage use by target population, (ii) discourage use by target population, (iii) encourage non-use by non-target population, (iv) discourage non-use by non-target population. This categorization is only meant to provide a systematic way to look at the data, in reality, many people fall at the intersection of these categories. The intention behind this form of classification is to help interventions identify what leads to non-use by the target population and improve their design to move towards their definition of use. In this case, we define 'use' to be the utilization of free government facilities by those who cannot afford quality health care over those who have access to affordable care.

\subsection{Factors/design discouraging non-use by non-target population}
Conversations with patients and doctors at the two Mohalla clinics revealed that the composition of the population visiting each clinic was different. While clinic A was visited mostly by people belonging to lower socioeconomic groups, the other, clinic B, was frequented by those belonging to the middle class. When the doctor at clinic B was informed that we were studying use of health care in low-resource environments, he suggested that we visit the other clinic, 

\textit{"More poor people visit the other Mohalla clinic. Here, most people you see are well-to-do."} 

Both clinics were located in the same geographical area, ten minutes from each other. Clinic A was located close to the slums along the Yamuna river while clinic B was near flats in a well-to-do neighborhood. %replace with another word

Most people learned of the clinic from neighbors or because they "lived here". The setting up of the clinics in that area was not announced formally, according to the doctor at clinic B, the government did not want more people coming to clinic than could handled. However, this meant that awareness of the clinics was limited to people living in its vicinity, and when that area was populated by middle-income groups, they were the ones who came to the clinic.

Before the setting up of clinic B, most people visiting the clinic would have would have visited a private facility nearby instead, while some would have visited the government dispensary. The option of free quality health care in proximity was a motivating factor in coming to the Mohalla clinics. 

More women came to the clinic than men, which could be because the clinics operated during the day when those working could not visit. An interesting phenomenon this lead to at clinic B was that the clinic functioned as a community center in once sense, a meeting point for women. Women would often meet women they knew and would be introduced to other women through them. They would come with their children and talk about their ailments and suggested home-made remedies they could try instead of allopathic medicines. Since the clinic was free, the women would go to the clinic for issues that they would not have visited a doctor for otherwise, ailments that did not chronically affect daily function such as mild back pain, rashes, and headaches. Many of those visiting the clinic were frequent visitors and could be seen at the clinic over multiple days while we were at the clinic. 

The proximity of the clinic, awareness of its presence, positive perception of health care and the safe and comfortable environment provided lead to people outside the target group visiting the clinic. However, this does not explain why the target group did not visit the clinics, which is discussed in the next section. An interesting phenomenon to note here is that awareness that the clinic was meant for the poor did not deter others from visiting the clinic. One woman's response when asked her opinion of the Mohalla clinic is mentioned below and was typical of other responses received, 

\textit{"I think the Mohalla clinic is a good initiative, it beings relief to the poor who cannot afford such facilities otherwise."}

During the course of the study, this phenomenon was not just observed at clinic B, one private hospital visited had separate hours for poor patients when the fees for consultation were lower. However, few poor patients could be seen during this time, most people visiting belonged to the middle-class since no documents were asked to demonstrate economic status.
%Location - Awareness (someone told them, neighbor, dr) - generally go to pvt - expensive - Free medicines - positive perception - Actors (implicitly? - preference to work in “better area” - comment) - Previous experience going to govt dispensary - Effective medicines - Patients come with multiple problems + kids - cannot do that with private - Perception of MC - Women talk while waiting - Waiting time
%\textbf{Quotes/Other data:}
%Patient's complain that medicines provided by private doctors aren't effective. 
%Knowing someone at the health facilities increases the confidence of the patients on the quality of care they are getting
%Some patients figure out a way to utilize the waiting time (esp if token system is in place)
%Poor patient OPD at the private hospital being utilized by middle class patients. 
%Free treatment and medicines means that patients come to the clinic for the smallest of health issues. Should this be deterred? How? Also, patients visit private clinics first and the come to MC to avail of free medicines/tests. Also, patients can dismiss the quality of the treatment because its free. 
\subsection{Factors/design discouraging use by target population}

*****

location, lack of info
religion
asking for documents
cultural factors
Also, patients can dismiss the quality of the treatment because it’s free.

%%CURRENT SITUATION
Lack of awareness
Cultural factors
Religion
Gender roles
Situation of women - Lack of support, abuse
Language barriers
Taboo on talking about sexual health
Hesitant talking about maternal care
timings
Lack of documents in certain cases (for other things)

\textbf{Quotes/Other data:}

Negative patient experiences and their effect on use of the health facilities. Eg: Complaint about potential vaccination side-effect. How do the dispensaries and MCs respond to patient's complaints? Would that affect the community's perception of the services they are getting (for eg: not up-to standards because the services are government run and free)

One of the ASHA workers reported community members tend to ignore her suggestions and has to force them. Why? Could it be due to lower levels of health care awareness? Maybe they don't see health playing an important role in terms of quality of life and their earning potential, how could this affect their health care seeking behavior? 

Housemaid seems to have good opinion of and detailed knowledge about the services available at government facilities. But also points out waiting time as a major barrier to accessing those services.

Patient's complain that medicines provided by private doctors aren't effective. 

Challenges unique to displaced populations (migrants living in slums)- no documents, long travel distance, high cost- deter them from accessing care.

ASHA worker's limited presence in slums 

Look for convo with maid: she talks about when she can go and her surprise at being unaware of the location of the clinic: Lack of advertising and publicly available information about locations/timings etc.  making the clinics hard to access.

\subsection{Factors/design encouraging use by target population}

*****

location
care
actors
Free treatment and medicines means that patients come to the clinic for the smallest of health issues. Should this be deterred? How? Also, patients visit private clinics first and then come to MC to avail of free medicines/tests. (quote)

%IDEAL SITUATION
Location
Awareness - Tell others - community structure (- ASHAs announcement at the mosque)
Affecting ability to work
Actors (spread awareness, imp of personal attention, force/motivate people)
Priority of dr
Religion - actors use religion to stress their point
Care for children (drives them to get care for kids in general)
Long waiting time
Care by drs/staff - abuse 
Need for empathy - place

\textbf{Quotes/Other data:}
Knowing someone at the health facilities increases the confidence of the patients on the quality of care they are getting

Role of religion? Religion of health workers and patients being the same might make the interactions easier.

Patient-doctor interactions: Atmosphere of comfort created by doctor being friendly, listening to life stories, berating like a concerned adult when necessary. What enables this? And can conditions be created to enable this? Is this even a good thing? MCs serving as places where a sane, authoritative person can hear out your problems (in some sense, as proxy community centers)

Mohalla clinics as places for enabling other health activities/initiatives e.g immunization drives.

Immunizations advertised through announcements in local mosques. Probably unrelated, but seems like an interesting medium for advertisement. Similar things could be done for MC but that could also be seen as advancing political agendas using religious institutions.

quote by ASHAs on "forcing people" and "threatening them with consequences of refusing immunization" (actors)

\subsection{Factors/design discouraging use by non-target population}
%IDEAL SITUATION

*****

Location - distance
Actors (implicitly and explicitly - what their expectations of the MC and their role are)

\textbf{Quotes/Other data:}
quotes by ASHAs and drs
\section{Discussion}
\textcolor{red}{Depends on what the nature of your study was, but you might either leave the reader with design recommendations for the context you studied (e.g., how could HCI address the lives of young women in rural West Bengal facing constant threats to their safety) or else discuss the main takeaways for your desired readers of this paper (ICTD researchers? Other HCI researchers? Practitioners?). This needs thought and work. You need a week just to think about and iterate on the Discussion. Do not take this lightly. If you are taking on a particular theoretical lens, you will need to deeply engage with this lens through the Discussion. 
}

How ethics play a role? How to still reach target

Class + socio-economic: look at references for class

Designing services and challenges involved, how to ensure services are adopted?


\section{Conclusion}
\textcolor{red}{[To be completed]}
%more here? 
The limited reach of the Mohalla clinics to its target group makes a case for viewing an intervention through the lens of designing for use and non-use by the target and non-target groups to ensure that it reaches those in 'need' of the intervention. While designing for use, the interplay of factors affecting use such as the design of the intervention, socioeconomic status, care by the actors and cultural norms should be taken into account. We offer this approach of 'designing for use and non-use' to the larger HCI4D community, as well as policy makers and organizations designing interventions.
\balance

\bibliographystyle{SIGCHI-Reference-Format}
\bibliography{references}

\end{document}
