\section{Findings}
\begin{comment}
This will form the chunk of your writing. I have sometimes iterated through these in written form, or else made a rough outline  on paper and then put it down. It is best that we talk about this section before you attempt to write. Depending on the structure of this section, you might label the section Findings or Analysis. The thing to remember is that all of your data goes into this section, and you need to weave a story around your data that is compelling, novel, and forms the central theme of the paper. If you have done qualitative work, make sure you include these quotes in the findings. Most importantly, do not tell a story based on quotes you have. Pick quotes based on the story you want to tell.

STORY OF USE
\end{comment}
\textcolor{red}{We present our findings from our field work.... [Rough categories - Story of USE?]}

\subsection{Challenges and Affordances (of accessing)}
sustainability - relying on goodwill - low pay + renting premises

\subsection{Design of the intervention}
Design - of the tablet/location/hours, could tie back to the challenges
timing: (open 9 - 1) but extend hours of clinic
guarantee seeing all patients
tablet: limited use - only recording the patient's details - what about using the data to improve the system, feedback system (that would go in discussion) (see pics of app)
location: 2 clinics - same geographical location - 10 minute walk from each other - one near the yamuna river where many of the slums are located. other flats

\subsection{Actors}
Actors - 3 layers - front-line health workers (ASHAs), doctors, patients (also seems to have an element of care)
govt perspective from news articles press releases. Missing piece of our 

Quotes and observations that may be relevant and/or interesting:
1. Look for convo with maid: she talks about when she can go and her surprise at being unaware of the location of the clinic (design)
Lack of advertising and publicly available information about locations/timings etc.  making the clinics hard to access.
2. pick one of the quotes on "live right here" (design)
3. quote by ASHAs on "forcing people" and "threatening them with consequences of refusing immunization" (actors)
4. religion - as a barrier - BUT - used by ASHAs as a tool to connect with ppl and further their outreach (actors?)
5. gender - barrier/facilitator in access? - how gender roles = improved/worse access to health? (affordances/actors?)
6. need for empathy - abuse - (actors) 
7. Quote on low pay (challenges?)
8. Patients come with multiple problems (affordances?)
9. Impact of MC (affordances)
10. Perceived care, Perception of MC (affordances/actors?)
11. Priority of dr, Expectations from doctors (actors)
12. Expectations of free, Expectations of MC (affordances?)
13. drs don't like to turn away patients (actors)
14. limitations of mobile phones in ASHAs work (design/actors?)
15. Waiting times at the clinic, physical token-based systems without information about the current number (which not be a standard across all MCs), while waiting patients chat with each other to pass time and this waiting time might be made more productive
16. Patients visiting MCs dont seem to keep medicines at home. In such scenario, MC might just play the role of a pharmacy supplying OTC medicines. Though the patients also see the doctors in the process, their ailments could be handled by easily available medicines. Lack of knowledge + the economic cost of buying extra medicines might be playing a role here.
17. Patients from any part of the city can visit any MC. Is this the best strategy esp. when the focus is on mohalla/neighborhood?
18. Weighing scale was a fascination for patients in the waiting area. Could similar, simple health tracking devices / posters be used in the waiting area to make the wait more fruitful?
19. Almost all patients report having heard of the MC through someone else (i.e. word of mouth propagation)
