\section{Findings}
\begin{comment}
This will form the chunk of your writing. I have sometimes iterated through these in written form, or else made a rough outline  on paper and then put it down. It is best that we talk about this section before you attempt to write. Depending on the structure of this section, you might label the section Findings or Analysis. The thing to remember is that all of your data goes into this section, and you need to weave a story around your data that is compelling, novel, and forms the central theme of the paper. If you have done qualitative work, make sure you include these quotes in the findings. Most importantly, do not tell a story based on quotes you have. Pick quotes based on the story you want to tell.

STORY OF USE
\end{comment}
\textcolor{red}{We present our findings from our field work.... [Rough categories - Story of Use?] At the moment, points below may be moved to discussion}

\subsection{Challenges and Affordances (of accessing)}
\textcolor{red}{sustainability - relying on goodwill - low pay + renting premises}

Patients complain that medicines provided by private doctors aren't effective.

Free treatment and medicines means that patients come to the clinic for the smallest of health issues. Should this be deterred? How? Also, patients visit private clinics first and then come to MC to avail of free medicines/tests. Also, patients can dismiss the quality of the treatment because it’s free. (there’s a quote on this)

Housemaid seems to have good opinion of and detailed knowledge about the services available at government facilities. But also points out waiting time as a major barrier to accessing those services.

Patients come with multiple problems 

low pay of doctors and staff

\subsection{Design of the intervention}
\textcolor{red}{Design - of the tablet/location/hours, could tie back to the challenges
timing: (open 9 - 1) but extend hours of clinic
guarantee seeing all patients
tablet: limited use - only recording the patient's details - what about using the data to improve the system, feedback system (that would go in discussion) (see pics of app)
location: 2 clinics - same geographical area - 10 minute walk from each other - one near the yamuna river where many of the slums are located. other - flats}

Immunizations advertised through announcements in local mosques. Probably unrelated, but seems like an interesting medium for advertisement. Similar things could be done for MC but that could also be seen as advancing political agendas using religious institutions.

Some patients figure out a way to utilize the waiting time (esp if token system is in place)

Waiting times at the clinic, physical token-based systems without information about the current number (which not be a standard across all MCs), while waiting patients chat with each other to pass time and this waiting time might be made more productive.

Patients visiting MCs dont seem to keep medicines at home. In such scenario, MC might just play the role of a pharmacy supplying OTC medicines. Though the patients also see the doctors in the process, their ailments could be handled by easily available medicines. Lack of knowledge + the economic cost of buying extra medicines might be playing a role here.

Patients from any part of the city can visit any MC. Is this the best strategy esp. when the focus is on mohalla/neighborhood?

Weighing scale was a fascination for patients in the waiting area (in every setting, while posters on diabetes, maternal health etc would go unnoticed). Could similar, simple health tracking devices / posters be used in the waiting area to make the wait more fruitful

Almost all patients report having heard of the MC through someone else (i.e. word of mouth propagation)

Most doctors working out of goodwill, not for money? Hiring process?

Whatsapp group for MC clinic doctors with health minister. Power of informal tech solutions?

Prompt response to doctor's complaints or requests for equipment. What sort of official channels used? - online forum, Are informal channels better?

Most patients are women. Why? Probably due to the timings which highlights a design flaw associated with the timings.

Mohalla clinics as places for enabling other health activities/initiatives e.g immunization drives

\subsection{Actors}
\textcolor{red}{
Actors - 3 layers - front-line health workers (ASHAs), doctors, patients (also seems to have an element of care)
govt perspective from news articles press releases. Missing piece of our study - govt's perspective}

One of the ASHA workers reported community members tend to ignore her suggestions and has to force them. Why? Could it be due to lower levels of health care awareness? Maybe they don't see health playing an important role in terms of quality of life and their earning potential, how could this affect their health care seeking behavior? - (this was also observed in the interactions with the ASHAs)

Patient-doctor interactions: Atmosphere of comfort created by doctor being friendly, listening to life stories, berating like a concerned adult when necessary. What enables this? And can conditions be created to enable this? Is this even a good thing? MCs serving as places where a sane, authoritative person can hear out your problems (in some sense, as proxy community centers)

Role of religion? Religion of health workers and patients being the same might make the interactions easier. 

doctor says - need for empathy over medicines - abuse

Uncategorized:
Challenges unique to displaced populations (migrants living in slums)- no documents, long travel distance, high cost- deter them from accessing care. 

Many health issues rising from unhygienic living conditions and otherwise preventable.

ASHA worker's limited presence in (SOME) slums

Yet to be organized:

1. Look for convo with maid: she talks about when she can go and her surprise at being unaware of the location of the clinic (design)
Lack of advertising and publicly available information about locations/timings etc.  making the clinics hard to access.

2. pick one of the quotes on "live right here" (design)

3. quote by ASHAs on "forcing people" and "threatening them with consequences of refusing immunization" (actors)

4. religion - as a barrier - BUT - used by ASHAs as a tool to connect with ppl and further their outreach (actors?)

5. gender - barrier/facilitator in access? - how gender roles = improved/worse access to health? (affordances/actors?)

6. need for empathy - abuse - (actors) 

7. Quote on low pay (challenges?)

8. Patients come with multiple problems (affordances?)

9. Impact of MC (affordances)

10. Perceived care, Perception of MC (affordances/actors?)

11. Priority of dr, Expectations from doctors (actors)

12. Expectations of free, Expectations of MC (affordances?)

13. drs don't like to turn away patients (actors)

14. limitations of mobile phones in ASHAs work (design/actors?)