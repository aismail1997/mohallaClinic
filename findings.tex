\section{Findings}%Analysis? 
%fix consistency: Clinics vs clinics. healthcare vs health care
%change current format - right now divided into subsections and subsubsections but may want to explore alternatives 
By observing and interviewing at clinics and select surroundings, we found that while the Mohalla Clinics had purportedly been designed for the ``weakest sections of the society'' \cite{AAPpressrelease}, these were not the sections that the Clinics were, in fact, catering to. In one of the clinics (referred to as Nooh
%change name - Nooh is the actual name of the area
 clinic henceforth), we learned from our interviews with the doctor and patients that the population largely consisted of individuals from the more privileged sections. However, at another clinic we visited that was located closer to the slums along the Yamuna river (which we refer to as Yamuna clinic), the population visiting the clinic was comprised primarily of people living in the slums and working as laborers, rickshaw drivers, scavengers and housemaids. Our fieldwork in nearby slums with the ``weakest sections'' indicated that there were many communities unaware of the presence of these clinics despite their proximity and did not avail these services. Driven %find a different word 
 by this seeming incongruence between the government's expectations and the ground reality - what were the factors that determined who attended these clinics then, we asked.% is this ok?

Below we highlight some of these factors that surfaced during our research. We first present our findings on `who' accessed these clinics and the reasons behind their decision to seek care. We then focus on those who pursued healthcare on a limited basis and the reasons for their lack of pursuit. We thus attempt to build a nuanced picture of the process of accessing healthcare by discussing the factors affecting use and non-use of these clinics by the target and non-target groups.
% go deeper about what access means
% Initial categories and content
\subsection{Who is targeted and who has access?}

As discussed previously, despite the government's explicit intention to target ``weakest sections'', our study showed that the two Mohalla clinic were accessed by different socioeconomic groups. We breakdown this phenomenon by looking at the  different themes that emerged from our data.

\subsubsection{Socioeconomic status}
First, we discuss how different actors involved with the clinics viewed the socioeconomic status of the patients coming to the clinic. % maybe remove this sentence
Press releases by the government[?] and news articles on the Mohalla clinics[?] were composed and disseminated, propagating the message that these clinics had been established for the benefit of the ``weakest sections of society''. Multiple patients coming to the clinic irrespective of their economic status would express their opinion of the clinic along the lines of this quote, \textit{``I think the Mohalla clinic is a good initiative, it brings relief to the poor who cannot afford such facilities otherwise.''}. 

Doctors, frontline health workers, and staff, also agreed that it was the poorest sections that the clinics had been set up for and were aware that the patient body at both the clinics was different. When the doctor at Nooh clinic was informed that we were studying use of health care in low-resource settings, he suggested that we visit the other clinic because, \textit{``More poor people visit the other Mohalla clinic. Here, most people you see are quite OK.''}

Those who were observed to be attending these Clinics were not the weakest sections. They belonged to different socioeconomic groups, for one. While Yamuna clinic, was visited mostly by people belonging to lower socioeconomic groups who lived in slums and worked as rickshaw drivers, housemaids, scavengers and manual laborers, Nooh clinic - in a market (???) area, was primarily frequented by those belonging to the \textit{middle class} - people who worked as shopkeepers, business owners, and clerks. However, this categorization of people into lower socioeconomic groups and the middle-class is not one that is clearly defined by the state or other actors and is up for interpretation. (add something here???) In our data from our interviews with the doctors, staff, and frontline workers, terms such as ``poor'', ``uneducated'', and ``laborers'' were frequently mentioned, and often used interchangeably, while describing or referring to patients. Additionally, patients would often use the same terms to describe themselves or others.

An interesting phenomenon to note here is that awareness that the clinic was targeted towards the poor did not deter non-target groups from visiting the clinic. During the course of the study, this phenomenon was not just observed at Nooh clinic, one private hospital visited had separate hours for poor patients called \textit{``poor patient OPD''} when the fees for consultation were around a hundred rupees lower. However, from our interview with the doctor and our observations, we found that few poor patients could be seen during this time, most people visiting belonged to the middle-class since no documents were asked to demonstrate economic status.

While the government did not explicitly state that those from the non-target group should not use the Mohalla clinics and did not design the clinics to discourage its use, implicit processes carried out by the doctors and front-line workers discouraged offering services to those belonging to the non-target group. Though, the doctor at the Yamuna clinic would not refuse to see patients outside the target group, he would express his displeasure when a patient came to the clinic asking for a certain medicine that he/she could afford to get from a pharmacy. For instance, he once told a patient he perceived as well-to-do who had come there expressly for medicines, \textit{``This clinic is not meant for people like you.''} % Quote

There were other background processes that determined who got care. The ANM  %should probably mention her earlier somewhere or remove
would tell the ASHAs to start work such as immunizations from the slums because, \textit{"Those in the flats can afford to get the vaccinations done from private clinics. For them, we have no value."} %unpack this statement further? 
 As a result, ASHAs would talk about family planning and maternal and neonatal care in the slums and would attend to pregnant women in lower socioeconomic groups over those who generally visited private facilities.

\subsubsection{Geographic location, proximity and existence of competing access points}
Being a market area, the location where our study was conducted had a host of private healthcare options, all right next to residential buildings. In addition to these, there was a government dispensary that provided free healthcare and since April 2016 %check date or maybe put aside all together??
, the Nooh and Yamuna Mohalla clinics were also providing free healthcare. Moving between these three government sources of health care was a matter of a 5 to 10 minute walk, something done by the ASHAs regularly to coordinate between the clincs and the dispensary.

Our interviews with the patients at the two clinics revealed that despite the whole area being easy to access on foot, %why do I say this?
most people visiting the clinics lived in close proximity to the clinic (such as the next building or the next lane). %fix this?
In the case of Nooh clinic, the area it was located in was mostly populated by the \textit{middle class}. As a result, most of the patients visiting the clinic belonged to relatively well-to-do backgrounds. % can we say as a result?
 On the other hand, the Yamuna clinic was located right next to the slums along the Yamuna and most of its patients were from these slums. In interviews, many of the patients irrespective of socioeconomic status would state that they knew about the clinic because they \textit{``live here''}. Before the setting up the Mohalla clinic, they said that they would have visited a private clinic that was also\textit{``right here''}. %maybe more detail on what right here means? - next bldg or lane

Though most people coming to the clinic lived closeby, there were also many people living close to the Mohalla clinics who did not know about their existence or had just found out about them. In one of the slums right next to the Yamuna clinic, a woman who worked as a housemaid was asked about the Mohalla clinic close to her home and she responded with surprise at its presence saying, \textit{``There is no government clinic near my place. Are you sure? This is near X? ... I did not hear of this in my community, no one mentioned it to me.''�}

Many people also found out about the clinic by chance because they happened to be walking past the building hosting the clinic and saw people come out with medicines. One woman who had found out about the Nooh clinic this way and was visiting for the first time told me, \textit{``I have been getting medicines worth 100 rupees daily for my son who cut his arm. We are poor and cannot afford the expense ... I live only a minute from here but I did not know about this place. I was just walking past this building today and saw people coming in. I asked someone what was going on and she told me that a clinic had opened up here. If I hadn't, I would have never known.''}

%Typical of other high density cities in India, the area had poor and relatively better off groups living in proximity.???

\subsubsection{Information available regarding competing access points}
%	Ppl found out from neighbors/live there/saw someone come out with medicines
% First - how is info generally disseminated in this area
% Second - what does the Mohalla clinic do
% Third - how do people disseminate info
Our interviews with the patients revealed that, before the setting up of the Mohalla clinics, most people would havevisited a private facility close to their home instead, while some would have visited the government dispensary, which was 5 minutes away but involved waiting for a couple of hours to receive service. The setting up of the Mohalla clinics provided people with an alternative that was closer, free, provided quality care and did not involve a long wait. 

Being a market area, advertisements for private clinics could be seen everywhere - on walls, e-rickshaws and banners. These advertisements provided information about the doctors, times of operation, the services available and its location. This pratice was in contrast to the Mohalla clinics that did not use this form of information dissemination at all, and instead relied heavily on word of mouth. Apart from news articles that talked about the clinics and their success and mentioned the locations of 1 or 2 clinics, there was no way to learn of the presence of a clinic. Even the clinics themselves were difficult to find in the absence of a crowd, Nooh clinic had a large poster of the clinic which was above the general field of vision, while Yamuna clinic just had a simple piece of paper with Mohalla clinic, and hours of operation written on it. %and imm date?

The setting up of the Mohalla clinics was not announced formally.Consequently, many of the patients visiting the clinics had found out about the clinic from someone else - a neighbor, a relative or a friend who had visited the clinic previously. According to the doctor at Yamuna clinic, the government was relying on word of moouth because they did not want more people coming to clinic than could be handled. However, this meant that awareness of the clinics was mostly limited to people living in its vicinity, and when that area was populated by middle-income groups, they were the ones who availed the clinics services despite not being part of the target group.

We looked more into how exactly information about the clinis was being shared via ``word of mouth'' and we learned from our interviews and observations that most of the patients visiting Yamuna clinic lived in slums in a close-knit community. The open structure of their dwellings facilitated free flow of information within that community and people would share information such as that of the presence of the Mohalla clinic with their neighbors. In the middle class communities visiting Nooh clinic, people tended to live in flats and women in particular would tell their neighbor about its presence who would tell a friend who would tell someone else and so on.

From our conversations with the ASHAs, we found that the frontline workers also had their own system for spreading information. In addition to personally telling people about the next immunization drive and health camp, announcements for these were made at the mosque near the clinic which spread awareness about the availability of free health care. Sometimes, the ASHAs would also tell the people about the Mohalla clinic when they were ill and were not able to visit a private clinic or a dispensary due to financial or time contraints.

%Since the opening of the Mohalla clinic, immunization drives organized by ASHAs were held here once a month. The Mohalla clinic also served as a point where people could inquire about when the next immunization drive was to be held as the clinics were in regular contact with the dispensary and the ASHAs operating in that area.

% Competing access - (Our interviews with the patients revealed that), Before the setting up of clinic B, most people from middle-class background visiting the clinic would have would have visited a private facility close to their home instead, while some would have visited the government dispensary (which was 5 minutes away). The setting up of the Mohalla clinics provided them with an alternative that was closer, free, provided quality care and did not involve a long wait. 

\subsubsection{Quality of care}
%Quotes on reception to clinic and perception of quality of care
%“Will come again”
%Quality of care, past experiences and return visits

All the people who came to clinic B said that they were satisfied with the quality care and one woman said,
%Quality of care
\textit{``The doctor treated me like I would have been treated at a private clinic.''}

While we did not observe any distinct difference in the use of clinical facilities based on socioeconomic status, what we did observe was that socio-cultural norms played a role in health seeking behavior. Our interactions with ASHAs brought out the role of these factors. In some communities, the impact of these factors on seeking health care in was so much so in areas such as immunization that ASHAs said that they had to resort to \textit{``forcing people''} and would threaten them with consequences for refusing immunization.

% medical conditions? - may want to remove this entirely
The medical conditions that the poor came to see a doctor for were different from the typical cases of cold, fever, throat pain and the like. The doctor at clinic A put it across thus,

\textit{``Since I've started working here, for the first time I'm seeing classic medical conditions that I have never seen in my 30 years of private practice and only saw in medical textbooks.''}

According to the doctor, the medical conditions that many patients at clinic A came with had to do with their living conditions and diet - infected scabies that manifested itself in the form of rashes, weakness and leg pain due to anemia caused by the presence of a worm in their body, and various vitamin deficiencies.

A staff member added,
\textit{``The people here have so many reports of irritation and rashes, I have never seen anything like this.''}

\subsubsection{Community feel}

Nooh - woman can talk to other women - kind of like a community center
Women discuss their problems and suggest alternative remedies

More women came to the clinic than men, which could be because the clinics operated during the day when those working could not visit. An interesting phenomenon this lead to at clinic Nooh was that the clinic functioned as a proxy community center, a meeting point for women. Women would often meet women they knew and would be introduced to other women through them. They would come with their children and discuss their ailments and suggest home-made remedies they could try instead of allopathic medicines. Since the clinic was free and convenient, the women would go to the clinic for issues that they would not have visited a doctor for otherwise, ailments that did not chronically affect daily function such as mild back pain, rashes, and headaches. Many of those visiting the clinic were frequent visitors and could be seen at the clinic over multiple days while we were at the clinic.

\subsubsection{Operation of the clinic}

%Timings

Additionally, the timings of the Mohalla clinics were not convenient for everyone. The clinics operated only during the day, during a time that many people work. In lower socioeconomic groups in this geographic area, the women were more likely to work to supplement their family's income. Thus, the very design of the clinics systematically determined who accessed them.

%geographical location
%waiting time
	More women come than men - more came later in the day
More women came to the clinic than men, which could be because the clinics operated during the day when those working could not visit.

Most of the patients visiting Yamuna clinic lived in slums in a close-knit community. The open structure of their dwellings facilitated free flow of information within that community and people would share information such as that of the presence of the Mohalla clinic with their neighbors. Additionally, announcements for immunization drives and health camps set up by the government were made at the mosque near the clinic which spread awareness about the availability of free health care. Since the opening of the Mohalla clinic, immunization drives organized by ASHAs were held here once a month. The Mohalla clinic also served as a point where people could inquire about when the next immunization drive was to be held as the clinics were in regular contact with the dispensary and the ASHAs operating in that area.

\subsubsection{Trust/Dr-patient relationship/care(?)}
Between dr and patient at Yamuna clinic - abuse - show bruises (Dr - “she moved her collar to show the mark of a hand”) 
Need for empathy quote by dr

The Mohalla clinics also provided a safe zone for women to talk about their situation at home such when facing physical abuse. They would show the bruises they had received to the doctor. In some ways, this filled an important gap in the support system available to women, one that was free from societal judgment. The doctor stated his view on this, 

\textit{``More than medical treatment, there is a need for empathy.''}

%education - may want to delete this entire para
The staff, doctors and front-line workers were of the view that the poor health of those coming to the clinic was directly or indirectly due to poor education, particularly that of the women. In many cases, the women had more than 3 children and had been married off early (before 18 years of age). Patients would ask the doctor and staff how to administer the medicines multiple times, and did not remember their age and in some cases, their address. As a result, the workers would approach the patients with an idea of what kind of information the patient would need and what they should stress on. Though this meant that they had preconceived notions of the population they were dealing with, in some ways this also facilitated care as they were intimately familiar with the context the patients were coming from and were careful about imparting the relevant information. (EXAMPLE?). According to a staff member,

\textit{``Most of the patients who come to this clinic have poor education...They aren’t knowledgeable about good health practices...''}

%care
The front-line workers also played an important role in filling the gap in the use of health care by encouraging and \textit{"motivating"} people to use health care facilities, including Mohalla clinics, and provided contraceptives, multivitamins and ORS. The ASHAs would berate those who did not immunize the children or use contraceptives and would express concern for the woman's health. For instance, an ASHA appealed to a woman who had already lost two children before the age of two to consider family planning saying that, 

\textit{``...It will be difficult to take care of two young children. If you want to live, use something (contraceptives). Your eyes are completely cloudy. For now, don't have more children for your health.''} % what does eyes are cloudy mean?

\subsection{
Lack of pursuit of health care}

\subsubsection{Religion}
%Cite religion as reason for not taking contraceptives
%ASHA - hindu areas - work is going well. People listen. Also because more educated
%ASHAs respond with - same religion, have read up on this

Some women would also refuse to use contraceptives citing religion as the reason saying that children were given by God. The ASHAs responded as follows,

\textit{``I belong to the same religion as you. I have read up on this topic and here's how it is - God will give you only what you are willing to take. If you are not agreeing to have more children (i.e. using contraceptives), then why will God give you more children?''}

Some women would also refuse to use contraceptives citing religion as the reason saying that children were given by God. The ASHAs responded as follows,

\textit{``I belong to the same religion as you. I have read up on this topic and here's how it is - God will give you only what you are willing to take. If you are not agreeing to have more children (i.e. using contraceptives), then why will God give you more children?''}

\subsubsection{Socio-cultural norms (?)}
%	Hesitant talking about sexual health
%	Hesitant talking about maternal health
%Refuse to immunize	

While we did not observe any distinct difference in the use of clinical facilities based on socioeconomic status, what we did observe was that socio-cultural norms played a role in health seeking behavior. Our interactions with ASHAs brought out the role of these factors. In some communities, the impact of these factors on seeking health care in was so much so in areas such as immunization that ASHAs said that they had to resort to \textit{``forcing people''} and would threaten them with consequences for refusing immunization.

Women in the communities visited were also hesitant talking about maternal care and sexual health. When offered contraceptives and asked to visit government hospitals for deliveries, those belonging to scavenger communities in particular would refuse. One ASHA pointed out a slum saying,

\textit{``No one here vaccinates their children. Also, the women here deliver in their homes. They give birth to the child then put him/her aside and get back to work.''} 

\subsubsection{Lack of independence of women/patriarchy}
%	Abuse
%lack of family support
%gender roles

%Interestingly, men from the same communities would immediately agree to take their wives to get an IUD inserted or get sterilized or take their wives to a hospital for delivery, though they would refuse to get sterilized themselves. While this phenomenon requires further investigation, it brings up the possibility of greater involvement of men in the process of spreading awareness about maternal and sexual health, which is particularly relevant when the skewed gender dynamic in "traditional" Indian contexts means that men have a greater say in their communities.

Our time at the Mohalla clinic also indicated that there may be deeper, more disturbing reasons behind why some women in particular may choose not to come to Mohalla clinics. Traditional gender roles in the context being studied mean that a woman has to bow down to her husband and his family's wishes. Refusal to do so could lead to abuse - both physical and mental. The doctor revealed that many of the women who came to the clinic had suffered physical abuse. One woman came to the Mohalla clinic to show her child who was severely malnourished to the extent that he could not walk despite being over two years old. When the doctor provided her with a medicine to stir into milk and give to the child, she burst into tears because she did not have money to buy milk. Later she shared this about her situation, \textit{``My husband beats me and doesn't give me money to see a doctor. I came here despite my husband telling me not to. If my child dies, then what?''}

%Lack of education - get married then sit at home: staff/ASHAs
%Ppl say: ladies should clean, cook, wash dishes - this is the result of that
%have kids early - affects health - ASHAs quote

%early marriage
The staff, doctors and front-line workers were of the view that the poor health of those coming to the clinic was directly or indirectly due to poor education, particularly that of the women. In many cases, the women had more than 3 children and had been married off early (before 18 years of age). Patients would ask the doctor and staff how to administer the medicines multiple times, and did not remember their age and in some cases, their address. As a result, the workers would approach the patients with an idea of what kind of information the patient would need and what they should stress on. Though this meant that they had preconceived notions of the population they were dealing with, in some ways this also facilitated care as they were intimately familiar with the context the patients were coming from and were careful about imparting the relevant information. (EXAMPLE?).

\textit{``A lot of these women got married young and have poor education, and have 4 or 5 or 6 kids ... last month there was a 13 year old girl who came with a 6 month old baby. This means she got married at the age of 12. If you come here every day, you will see patients like this every once in a while.''}

\subsubsection{Language barriers}
%Migrants - Bengali community of scavengers - difficult to communicate have to talk through ppl in community who work outside the home and know hindi (mostly men)
ASHA says “will do this section another day” after gets tired talking to people. 
They refuse immunization and give birth at home - many had around 6 children.

Additionally, most of the members of the scavenger community knew only Bengali which none of the ASHAs had any knowledge of. This created an additional barrier to health care - one of language. One ASHA pointed out a slum saying,

\textit{"No one here vaccinates their children. Also, the women here deliver in their homes. They give birth to the child then put him/her aside and get back to work."} 


%First, who is being targeted? Newspaper reports that were (really publicity material -?)% can we write that?
%for the Clinics were composed and disseminated, propagating the message that these clinics had been established for the benefit of the ``weakest sections of society''. Doctors, staff, others???, also agreed that it was the poorest sections that these Clinics had been set up for. Those who were observed to be attending these Clinics were not the weakest sections. They belonged to different socioeconomic groups, for one. While clinic A, next to the Yamuna, was visited mostly by people belonging to lower socioeconomic groups who lived in slums and worked as rickshaw drivers, housemaids, scavengers and manual laborers, clinic B - in X area, was primarily frequented by those belonging to the middle class - people who worked as shopkeepers, business owners, and clerks. However, this category is not one that is clearly defined by the state or other actors and was up for interpretation.

%Conversations with patients and doctors at the two Mohalla clinics revealed that those visiting the clinics belonged to different socioeconomic groups. While clinic A was visited mostly by people belonging to lower socioeconomic groups who lived in slums and worked as rickshaw drivers, housemaids, scavengers and manual laborers, clinic B, was primarily frequented by those belonging to the middle class - people who worked as shopkeepers, business owners, and clerks. When the doctor at clinic B was informed that we were studying use of health care in low-resource environments, he suggested that we visit the other clinic because,
%\textit{"More poor people visit the other Mohalla clinic. Here, most people 
%you see are quite OK."}

 %perception of govt health care facilities - Can mention this as well. Don't know if it will add value though
%However, apart from the role of the actors and the absence of a clinic in the proximity of their home, little else actively discouraged people belonging to non-target groups from visiting the clinics.

%\textbf{\textit{Class}}

%“Middle class and lower class” - who? - according to whom?
%ASHAs - perception of class - educated, have money, have only 1-2 children, big homes
%Middle class - some dr’s/health workers discourage use by them (for them we have no value)

%\textbf{\textit{Who is poor?}}

%Acc to drs, frontline workers, patients: quotes - “don’t have money”, “gareeb”, “mazdoor”, “cannot afford”, “padhe-likhe nahin hain” 
%Poor patient opd at private hospital - used mostly by middle class - dr says + observation
%Staff: Educated take medicines when we need them using judgement - poor ppl don’t, (dr - having medicines at home is like a bad omen)
