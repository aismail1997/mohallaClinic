\section{Findings}
\begin{comment}
This will form the chunk of your writing. I have sometimes iterated through these in written form, or else made a rough outline  on paper and then put it down. It is best that we talk about this section before you attempt to write. Depending on the structure of this section, you might label the section Findings or Analysis. The thing to remember is that all of your data goes into this section, and you need to weave a story around your data that is compelling, novel, and forms the central theme of the paper. If you have done qualitative work, make sure you include these quotes in the findings. Most importantly, do not tell a story based on quotes you have. Pick quotes based on the story you want to tell.

STORY OF USE
\end{comment}
\textcolor{red}{[Rudimentary subheadings]. We present our findings from our field work....}

\subsection{Challenges and Affordances}
challenges/affordances - of accessing - look at data:


\subsection{Design of the intervention}
Design - of the tablet/location/hours, could tie back to the challenges


\subsection{Actors}
Actors - 3 layers - front-line health workers, doctors, patients
\langle{"Around 2 years ago, I used to visit a private hospital in Okhla, but it was expensive...I use to hesitate to use government facilities earlier because the facilities were free and I thought that they might not be good. But one time I went to the dispensary after I did not feel better after going to a private doctor. I had to wait in line for a long time to see a doctor and I even contemplated leaving. Finally, I got to see the doctor and he gave me just one pill for my problem unlike the multiple medicines I had received before. And I felt better immediately after that. Ever since then, I’ve only used government facilities."}
\langle{}



