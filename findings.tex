\section{Findings}
\textcolor{red}{}
%In this section, we discuss our findings about the design and implementation of Mohalla clinics. In particular, we focus on factors that affect the use of these clinics by the target population and the non-use of these clinics by the non-target population. We consider, both, factors that encourage and discourage the desired behaviors. Thus, we classify our findings into four different categories. Factors that: (i) encourage use by target population, (ii) discourage use by target population, (iii) encourage non-use by non-target population, (iv) discourage non-use by non-target population.
%(RM: Assuming target population is folks from lower socio-economic background and non-target population is everyone else i.e. folks from higher socio-economic background.  Will have mentioned this in the intro.) 
Our analysis of the data collected revealed a story of use around Mohalla clinics and other government health care services. Other themes of gender, religion, education, socioeconomic and cultural factors emerged from our data around the larger theme of use and non-use. 
% A = Yamuna, B = ?
To discuss these themes in the context of use, we classify our findings into four different categories - factors that: (i) encourage use by target population, (ii) discourage use by target population, (iii) encourage non-use by non-target population, (iv) discourage non-use by non-target population. (This categorization is only meant to provide a systematic way to look at the data, in reality,) many people fall at the intersection of these categories. The intention behind this form of classification is to help interventions identify what leads to non-use by the target population and improve their design to move towards their definition of use. In the case of the Mohalla Clinics, we define 'use' to be the utilization of free government facilities by those who cannot afford quality health care over those who have access to affordable care.
% go deeper about what access means
% REMINDER - always mentuin 

\subsection{Factors/design discouraging non-use by non-target population}
Conversations with patients and doctors at the two Mohalla clinics revealed that those visiting the clinics belonged to different socioeconomic groups. While clinic A was visited mostly by people belonging to lower socioeconomic groups who lived in slums and worked as rickshaw drivers, housemaids, scavengers and manual laborers, clinic B, was primarily frequented by those belonging to the middle class - people who worked as shopkeepers, business owners, and clerks. When the doctor at clinic B was informed that we were studying use of health care in low-resource environments, he suggested that we visit the other clinic because,

\textit{"More poor people visit the other Mohalla clinic. Here, most people 
you see are well-to-do."}    
% 

Both clinics were located in the same geographical area, only ten minutes walk 5 minutes from each other. Clinic A was located close to the slums along the Yamuna river while clinic B was situated near the flats in a well-to-do neighborhood. % replace with another word? - theek thak?

Most people learned of the clinic from neighbors or because the  \textit{"lived right here"}. The setting up of the clinics in that area was not announced formally, and news of its presence spread by word of mouth. According to the doctor at clinic B, the government did not want more people coming to clinic than could be handled. However, this meant that awareness of the clinics was mostly limited to people living in its vicinity, and when that area was populated by middle-income groups, they were the ones who came to the clinic.

Before the setting up of clinic B, most people from middle-class background visiting the clinic would have would have visited a private facility close to their home instead, while some would have visited the government dispensary (which was 5 minutes away). The setting up of the Mohalla clinics provided them with an alternative that was closer, free, provided quality care and did not involve a long wait. All the people who came to clinic B said that they were satisfied with the quality care and one woman said,

\textit{"The doctor treated me like I would have been treated at a private clinic."}

More women came to the clinic than men, which could be because the clinics operated during the day when those working could not visit. An interesting phenomenon this lead to at clinic B was that the clinic functioned as a proxy community center, a meeting point for women. Women would often meet women they knew and would be introduced to other women through them. They would come with their children and discuss their ailments and suggest home-made remedies they could try instead of allopathic medicines. Since the clinic was free, the women would go to the clinic for issues that they would not have visited a doctor for otherwise, ailments that did not chronically affect daily function such as mild back pain, rashes, and headaches. Many of those visiting the clinic were frequent visitors and could be seen at the clinic over multiple days while we were at the clinic. 

The proximity of the clinic, awareness of its presence, positive perception of health care and the safe and comfortable environment provided lead to people outside the target group visiting the clinic. However, this does not explain why the target group did not visit the clinics, which is discussed in the next section. 

An interesting phenomenon to note here is that awareness that the clinic was targeted towards the poor did not deter others from visiting the clinic. One woman's response when asked her opinion of the Mohalla clinic is mentioned below and was typical of other responses received, 

\textit{"I think the Mohalla clinic is a good initiative, it beings relief to the poor who cannot afford such facilities otherwise."}

During the course of the study, this phenomenon was not just observed at clinic B, one private hospital visited had separate hours for poor patients when the fees for consultation were lower. However, few poor patients could be seen during this time, most people visiting belonged to the middle-class since no documents were asked to demonstrate economic status.

\subsection{Factors/design discouraging use by target population} %perhaps discourage isn't the best word in this case - disabling may work better
At the surface level, lack of awareness about the Mohalla clinic was one of the primary reasons that the target group did not access the Mohalla Clinics. At clinic B, those belonging to lower economic groups who did visit had found out about this clinic because they happened to see someone come out of the gate with medicines or saw a large crowd going in. One woman had been getting medicines for her children regularly which was putting a strain on her already meager income. 

\textit{"I have been getting medicines worth 100 rupees daily for my son who cut his arm. We are poor and cannot afford the expense ... I live only a minute from here but I did not know about this place. I was just walking past this building today and saw people coming in. I asked someone what was going on and she told me that a clinic had opened up here. If I hadn't, I would have never known. If the doctor treats me well then I will show my children here."}

There were also those living near clinic A who did not know about its presence. In one of the slums, a woman who worked as a housemaid was asked about the Mohalla clinic close to her home and she responded with surprise at its presence,

\textit{"There is no government clinic near my place. Are you sure? This is near X? ... I did not hear of this in my community, no one mentioned it to me. I have been ill recently and my children have rashes so I will visit the doctor...”}

Additionally, the timings of the Mohalla clinics were not convenient for everyone. The clinics operated only during the day, during a time that many people work. In lower socioeconomic groups in this geographic area, the women were more likely to work to supplement their family's income. Thus, the very design of the clinics systematically determined who accessed them.

Going beyond design factors, deeper questions regarding the use of health care in general by those belonging to lower socioeconomic groups arise. Is there a fundamental difference in how the poor make decisions regarding seeking care? If so, this affects the success of the intervention and needs to be addressed.

While we did not observe any distinct difference in the use of clinical facilities based on socioeconomic status, what we did observe was that socio-cultural norms played a role in health seeking behavior. Our interactions with ASHAs brought out the role of these factors. In some communities, the impact of these factors on seeking health care in was so much so in areas such as immunization that ASHAs said that they had to resort to \textit{"forcing people"} and would threaten them with consequences for refusing immunization.

Women in the communities visited were also hesitant talking about maternal care and sexual health. When offered contraceptives and asked to visit government hospitals for deliveries, those belonging to scavenger communities in particular would refuse. Additionally, most of the members of the scavenger community knew only Bengali which none of the ASHAs had any knowledge of. This created an additional barrier to health care - one of language. One ASHA pointed out a slum saying,

\textit{"No one here vaccinates their children. Also, the women here deliver in their homes. They give birth to the child then put him/her aside and get back to work."} 

Interestingly, men from the same communities would immediately agree to take their wives to get an IUD inserted or get sterilized or take their wives to a hospital for delivery, though they would refuse to get sterilized themselves. While this phenomenon requires further investigation, it brings up the possibility of greater involvement of men in the process of spreading awareness about maternal and sexual health, which is particularly relevant when the skewed gender dynamic in "traditional" Indian contexts means that men have a greater say in their communities. %perhaps a citation here?

Some women would also refuse to use contraceptives citing religion as the reason saying that children were given by God. The ASHAs responded as follows,

\textit{"I belong to the same religion as you. I have read up on this topic and here's how it is - God will give you only what you are willing to take. If you are not agreeing to have more children (i.e. using contraceptives), then why will God give you more children?"}

Our time at the Mohalla clinic also indicated that there may be deeper, more disturbing reasons behind why some women in particular may choose not to come to Mohalla clinics. Traditional gender roles in the context being studied mean that a woman has to bow down to her husband and his family's wishes. Refusal to do so could lead to abuse - both physical and mental. The doctor revealed that many of the women who came to the clinic had suffered physical abuse. One woman came to the Mohalla clinic to show her child who was severely malnourished to the extent that he could not walk despite being over two years old. When the doctor provided her with a medicine to stir into milk and give to the child, she burst into tears because she did not have money to buy milk. Later she shared this about her situation, \textit{"My husband beats me and doesn’t give me money to see a doctor. I came here despite my husband telling me not to. If my child dies, then what?"}

\subsection{Factors/design encouraging use by target population}
Most of the patients visiting clinic A lived in slums in a close-knit community. The open structure of their dwellings facilitated free flow of information within that community and people would share information such as that of the presence of the Mohalla clinic with their neighbors. Additionally, announcements for immunization drives and health camps set up by the government were made at the mosque near the clinic which spread awareness about the availability of free health care. Since the opening of the Mohalla clinic, immunization drives organized by ASHAs were held here once a month. The Mohalla clinic also served as a point where people could inquire about when the next immunization drive was to be held as the clinics were in regular contact with the dispensary and the ASHAs operating in that area.

While the proximity of the clinic to the dwellings of the target group and their awareness of its presence were factors that lead to the use of the facilities by the target group, at a very basic level, the reason the target group came to the clinic was the need for a quality health care when previously none was available or affordable. Our data also shows that the needs of those from lower socioeconomic groups visiting the clinic were different in some ways which impacted their experience accessing health care at the Mohalla clinics, as well as that of the staff's, and we discuss this further in this section. These arise from their poor living environment, lack of nutritious diet, poor education, and societal norms. The Mohalla Clinics potentially fulfill some of these needs.

The medical conditions that the poor came to see a doctor for were different from the typical cases of cold, fever, throat pain and the like. The doctor at clinic A put it across thus,

\textit{"Since I've started working here, for the first time I'm seeing classic medical conditions that I have never seen in my 30 years of private practice and only saw in medical textbooks."}

According to the doctor, the medical conditions that many patients at clinic A came with had to do with their living conditions and diet - infected scabies that manifested itself in the form of rashes, weakness and leg pain due to anemia caused by the presence of a worm in their body, and various vitamin deficiencies.

A staff member added,
\textit{"The people here have so many reports of irritation and rashes, I have never seen anything like this."}

The Mohalla clinics also provided a safe zone for women to talk about their situation at home such when facing physical abuse. They would show the bruises they had received to the doctor. In some ways, this filled an important gap in the support system available to women, one that was free from societal judgment. The doctor stated his view on this, 

\textit{"More than medical treatment, there is a need for empathy."}

The staff, doctors and front-line workers were of the view that the poor health of those coming to the clinic was directly or indirectly due to poor education, particularly that of the women. In many cases, the women had more than 3 children and had been married off early (before 18 years of age). Patients would ask the doctor and staff how to administer the medicines multiple times, and did not remember their age and in some cases, their address. As a result, the workers would approach the patients with an idea of what kind of information the patient would need and what they should stress on. Though this meant that they had preconceived notions of the population they were dealing with, in some ways this also facilitated care as they were intimately familiar with the context the patients were coming from and were careful about imparting the relevant information. According to a staff member,

\textit{“Most of the patients who come to this clinic have poor education...They aren’t knowledgeable about good health practices..."}

\textit{"A lot of these women got married young and have poor education, and have 4 or 5 or 6 kids ... last month there was a 13 year old girl who came with a 6 month old baby. This means she got married at the age of 12. If you come here every day, you will see patients like this every once in a while.”}

The front-line workers also played an important role in filling the gap in the use of health care by encouraging and \textit{"motivating"} people to use health care facilities, including Mohalla clinics, and provided contraceptives, multivitamins and ORS. The ASHAs would berate those who did not immunize the children or use contraceptives and would express concern for the woman's health. For instance, an ASHA appealed to a woman who had already lost two children before the age of two to consider family planning saying that, 

\textit{"...It will be difficult to take care of two young children. If you want to live, use something (contraceptives). Your eyes are completely cloudy. For now, don’t have more children for your health.}

\subsection{Factors/design discouraging use by non-target population}
While the government did not explicitly state that those from the non-target group should not use the Mohalla clinics and did not design the clinics to discourage its use, there were implicit processes carried out by the doctors and front-line workers that discouraged offering services to those belonging to the non-target group. Though, the doctor at clinic A would not refuse to see patients outside the target group, he would express his displeasure when a patient came to the clinic asking for a certain medicine that he/she could afford to get from a pharmacy.

There were other background processes that determined who got care. The ANM would tell the ASHAs to start work such as immunizations from the slums because, \textit{"Those in the flats can afford to get the vaccinations done from private clinics. For them, we have no value."} As a result, ASHAs would talk about family planning and maternal and neonatal care in the slums and would attend to pregnant women in lower socioeconomic groups over those who generally visited private facilities. %perception of govt health care facilities - Can mention this as well. Don't know if it will add value though

However, apart from the role of the actors and the absence of a clinic in the proximity of their home, little else actively discouraged people belonging to non-target groups from visiting the clinics.