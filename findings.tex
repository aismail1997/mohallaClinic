\section{Findings}
\begin{comment}
This will form the chunk of your writing. I have sometimes iterated through these in written form, or else made a rough outline  on paper and then put it down. It is best that we talk about this section before you attempt to write. Depending on the structure of this section, you might label the section Findings or Analysis. The thing to remember is that all of your data goes into this section, and you need to weave a story around your data that is compelling, novel, and forms the central theme of the paper. If you have done qualitative work, make sure you include these quotes in the findings. Most importantly, do not tell a story based on quotes you have. Pick quotes based on the story you want to tell.
STORY OF USE
\end{comment}
\textcolor{red}{}
%In this section, we discuss our findings about the design and implementation of Mohalla clinics. In particular, we focus on factors that affect the use of these clinics by the target population and the non-use of these clinics by the non-target population. We consider, both, factors that encourage and discourage the desired behaviors. Thus, we classify our findings into four different categories. Factors that: (i) encourage use by target population, (ii) discourage use by target population, (iii) encourage non-use by non-target population, (iv) discourage non-use by non-target population. 
%(RM: Assuming target population is folks from lower socio-economic background and non-target population is everyone else i.e. folks from higher socio-economic background.  Will have mentioned this in the intro.) 
Our analysis of the data collected revealed a story of use around Mohalla clinics and other government health care services. Other themes of gender, religion, education, socioeconomic and cultural factors emerged from our data, these revolve around the larger theme of use and non-use. To discuss these themes in the context of use, we classify our findings into four different categories - factors that: (i) encourage use by target population, (ii) discourage use by target population, (iii) encourage non-use by non-target population, (iv) discourage non-use by non-target population. This categorization is only meant to provide a systematic way to look at the data, in reality, many people fall at the intersection of these categories. The intention behind this form of classification is to help interventions identify what leads non-use by the target population and improve their design to move towards their definition of use. In this case, we define 'use' to be the utilization of free government facilities by those who cannot afford health care.

%(why do we choose this classification?)
\subsection{Factors/design discouraging non-use by non-target population}

%CURRENT SITUATION
Conversations with patients and doctors at the two Mohalla clinics revealed that the composition of the population visiting each clinic was different. While one clinic was primarily frequented by people belonging to lower socioeconomic groups, the other was frequented by those belonging to the middle class. When the doctor at one clinic was informed that we were studying use of health care in low-resource environments, he suggested that we visit the other clinic, 

\textit{"More poor people visit the other Mohalla clinic. Here, most people you see are well-to-do."} 

While both clinics were located in the same geographical area ten minutes from each other, one of the clinics was located close to the slums along the Yamuna river while the other was near flats in a well to do area. %replace with another word



*****

Location
Awareness (someone told them, neighbor, dr)
generally go to pvt - expensive
Free medicines - positive perception
Actors (implicitly? - preference to work in “better area” - comment)
Previous experience going to govt dispensary
Effective medicines
Patients come with multiple problems +kids - cannot do that with private
Perception of MC
Women talk while waiting
Waiting time

\textbf{Quotes/Other data:}
Patient's complain that medicines provided by private doctors aren't effective. 

Knowing someone at the health facilities increases the confidence of the patients on the quality of care they are getting
Some patients figure out a way to utilize the waiting time (esp if token system is in place)

Poor patient OPD at the private hospital being utilized by middle class patients. 

Free treatment and medicines means that patients come to the clinic for the smallest of health issues. Should this be deterred? How? Also, patients visit private clinics first and the come to MC to avail of free medicines/tests. Also, patients can dismiss the quality of the treatment because its free. 

\subsection{Factors/design discouraging use by target population}

*****

location, lack of info
religion
asking for documents
cultural factors
Also, patients can dismiss the quality of the treatment because it’s free.

%%CURRENT SITUATION
Lack of awareness
Cultural factors
Religion
Gender roles
Situation of women - Lack of support, abuse
Language barriers
Taboo on talking about sexual health
Hesitant talking about maternal care
timings
Lack of documents in certain cases (for other things)

\textbf{Quotes/Other data:}

Negative patient experiences and their effect on use of the health facilities. Eg: Complaint about potential vaccination side-effect. How do the dispensaries and MCs respond to patient's complaints? Would that affect the community's perception of the services they are getting (for eg: not up-to standards because the services are government run and free)

One of the ASHA workers reported community members tend to ignore her suggestions and has to force them. Why? Could it be due to lower levels of health care awareness? Maybe they don't see health playing an important role in terms of quality of life and their earning potential, how could this affect their health care seeking behavior? 

Housemaid seems to have good opinion of and detailed knowledge about the services available at government facilities. But also points out waiting time as a major barrier to accessing those services.

Patient's complain that medicines provided by private doctors aren't effective. 

Challenges unique to displaced populations (migrants living in slums)- no documents, long travel distance, high cost- deter them from accessing care.

ASHA worker's limited presence in slums 

Look for convo with maid: she talks about when she can go and her surprise at being unaware of the location of the clinic: Lack of advertising and publicly available information about locations/timings etc.  making the clinics hard to access.

\subsection{Factors/design encouraging use by target population}

*****

location
care
actors
Free treatment and medicines means that patients come to the clinic for the smallest of health issues. Should this be deterred? How? Also, patients visit private clinics first and then come to MC to avail of free medicines/tests. (quote)

%IDEAL SITUATION
Location
Awareness - Tell others - community structure (- ASHAs announcement at the mosque)
Affecting ability to work
Actors (spread awareness, imp of personal attention, force/motivate people)
Priority of dr
Religion - actors use religion to stress their point
Care for children (drives them to get care for kids in general)
Long waiting time
Care by drs/staff - abuse 
Need for empathy - place

\textbf{Quotes/Other data:}
Knowing someone at the health facilities increases the confidence of the patients on the quality of care they are getting

Role of religion? Religion of health workers and patients being the same might make the interactions easier.

Patient-doctor interactions: Atmosphere of comfort created by doctor being friendly, listening to life stories, berating like a concerned adult when necessary. What enables this? And can conditions be created to enable this? Is this even a good thing? MCs serving as places where a sane, authoritative person can hear out your problems (in some sense, as proxy community centers)

Mohalla clinics as places for enabling other health activities/initiatives e.g immunization drives.

Immunizations advertised through announcements in local mosques. Probably unrelated, but seems like an interesting medium for advertisement. Similar things could be done for MC but that could also be seen as advancing political agendas using religious institutions.

quote by ASHAs on "forcing people" and "threatening them with consequences of refusing immunization" (actors)

\subsection{Factors/design discouraging use by non-target population}
%IDEAL SITUATION

*****

Location - distance
Actors (implicitly and explicitly - what their expectations of the MC and their role are)

\textbf{Quotes/Other data:}
quotes by ASHAs and drs