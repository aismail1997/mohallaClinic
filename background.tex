%\section{Background}
\begin{comment}
Build up the context that the reader could read, like a story, to understand more about the problem you’re addressing, the phenomena you are talking about. If the paper is about an internet ban in Bangladesh, tell the story about how the ban was put in place, what was the timeline like, who said what in popular press, and so on. If the research was conducted in the context of an organization of health/outreach workers, talk about the organization, its charter, what it has done/is doing.

\textcolor{red}{ [
health scenario in Delhi - current system - proposed system by govt - current implementation of proposed system - questions around the proposed system]
}

Despite advances made in public policy and infrastructure in health care in Delhi, results have been mixed. While having the lowest death rates and one of the best life expectancy levels in the country, infant mortality is high and there is a significant gap in the uptake of maternal and child care among slum and non-slum populations. Given the significantly better availability of health facilities as compared to other cities, limited success in areas such as institutional deliveries suggests the persistence of socioeconomic inequalities in access \cite{mazumdar2015health}.

To cover the last mile in health care, the Delhi government has proposed a three tier system that supports the existing network of government hospitals and dispensaries that offer free health care. \cite{article}. At the lowest tier of the proposed system are neighborhood clinics (or Mohalla clinics) that are operated by one doctor assisted by staff and offer free consultations, medication and over 200 tests. At the secondary level, polyclinics will be set up that provide specialized services. The existing large government hospitals make up the top-most level of the system.

While the existing dispensaries are meant to cater to 50,000 people each, one Mohalla Clinic is meant to cater to around 10,000 people. As of August 2016, just over a hundred Mohalla Clinics and 2 polyclinics have been set up. The clinics have been set up in rented rooms around the city and [infrastructure provided - operation time -  who said what in popular press - background of the doctors - pay - presence of staff] \cite{articles}. Our study focuses only on the lowest tier of the system - the Mohalla Clinics, and the questions that arise from its use.

\end{comment}